
\begin{deluxetable*}{lcc}

\tabletypesize{\scriptsize}
\tablewidth{0pt}
\tablecaption{\label{tab:sns05} The Timing Results from 
\citealt{sns+05} and from a Re-analysis of the \cite{sns+05} Data set
Using New Red Noise Analysis Technique.}
\tablehead{ \colhead{Parameter}  &\colhead{\citealt{sns+05}}  &\colhead{Red Noise Model}\tablenotemark{a}  }
\startdata
%\textit{Measured Parameters}&  &  \\
R. A., $\alpha$ (J2000)&  17:13:49.5305335(6)&  17:13:49.5305321(6)\\
Decl., $\delta$ (J2000)&  7:47:37.52636(2)&  7:47:37.52626(2)\\
Spin frequecy $\nu$~(s$^{-1}$)&  218.8118439157321(3)& 218.811843915731(1)\\
Spin down rate $\dot{\nu}$ (s$^{-2}$)&  $-4.0835(2)\times10^{-16}$&  $-4.0836(1)\times10^{-16}$\\
Proper motion in $\alpha$, $\mu_{\alpha}=\dot{\alpha}\cos \delta$ (mas~yr$^{-1}$)&  4.917(4)&  4.917(4)\\
Proper motion in $\delta$, $\mu_{\delta}=\dot{\delta}$ (mas~yr$^{-1}$)& $-$3.93(1)&  $-$3.93(1)\\
Parallax, $\varpi$ (mas)&  0.89(8)&  0.84(4)\\
Dispersion measure (pc~cm$^{-3}$)&  15.9960&  15.9940\\
Orbital period, $P_{\rm b}$ (day)&  67.8251298718(5)\tablenotemark{b}  &  67.825129921(4)\\
Change rate of $P_{\rm b}$, $\dot{P}_{\rm b}$ ($10^{-12}$s~s$^{-1}$)& 0.0(6)&  $-$0.2(7)\\
Eccentricity, $e$&  0.000074940(1)\tablenotemark{b}  &  0.000074940(1)\\
Time of periastron passage, $T_0$ (MJD)&  51997.5784(2) \tablenotemark{b} & 51997.5790(6)\\
Angle of periastron, $\omega$ (deg)&  176.192(1)\tablenotemark{b} &  176.195(3)\\
Projected semi-major axis, $x$ (lt-s)&  32.34242099(2)\tablenotemark{b} &  32.3424218(3)\\
Cosine of inclination, $\cos i$&  0.31(3)&  0.32(2)\\
Companion mass, $M_{\rm c}$ ($M_{\odot}$)&  0.28(3)&  0.30(3)\\
Position angle of ascending node, $\Omega$ (deg)&  87(6)&  89(4)\\
%Rate of periastron advance, $\dot{\omega}$ (deg/yr)&  0.0007(4)&  0.0004(4)\\
%\textit{Fixed Parameters}&  &  \\
Solar system ephemeris&  DE405&  DE405\\
Reference epoch for $\alpha$, $\delta$, and $\nu$ (MJD)&  52000&  52000\\
Pulsar mass, $M_{\rm PSR}$ ($M_{\odot}$)&  1.3(2)&  1.4(2)\\
Red noise amplitude ($\mu$s~${\rm yr}^{1/2}$)&  --&  0.004\\
Red noise spectral index&  --&  5.14\\
%\textit{Derived Parameters}&  &  \\
%Intrinsic period derivative, $\dot{P}_{\rm Int}$(s~s$^{-1}$)&  $8.95(13)\times10^{-21}$&  $8.98(3)\times10^{-21}$\\
%Dipole magnetic field, $B$ (G)&  $2.046(15)\times10^{+8}$&  $2.050(4)\times10^{+8}$\\
%Characteristic age, $\tau_c$ (yr)&  $8.09(12)\times10^{+9}$&  $8.07(3)\times10^{+9}$
\enddata
\tablenotetext{a}{We used {\sc TEMPO2}'s {\it T2} binary model, which models
the Keplerian ($P_{\rm b}$, $x$, $e$, $T_0$, and $\omega$) and
post-Keplerian orbital elements ($\cos i$, $\Omega$, and $m_2$ ) simultaneously.}
\tablenotetext{b}{\citealt{sns+05} uses {\sc TEMPO}'s DD model and reports the uncertainties of the Keplerian
parameters with the post-Keplerian ones fixed to their bestfit values. }
\end{deluxetable*}

