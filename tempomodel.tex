
%\clearpage
\begin{deluxetable*}{lccc}

\tabletypesize{\scriptsize}
\tablewidth{0pt}
\tablecaption{\label{tab:par1} Timing model parameters\tablenotemark{a} from
{\it TEMPO}. }
\tablehead{ \colhead{Parameter}  &\colhead{EFAC \& EQUAD}  &\colhead{with jitter model}  &\colhead{jitter \& red noise model}   }
\startdata
\textit{Measured Parameters}&  &  &  \\%[1mm]
Right Ascension, $\alpha$ (J2000)&  17:13:49.5320251(5)&  17:13:49.5320248(7)&  17:13:49.5320252(8)\\
Declination, $\delta$ (J2000)&  7:47:37.506150(12)&  7:47:37.506155(19)&  7:47:37.50614(2)\\
Spin Frequecy $\nu$~(s$^{-1}$)&  218.81184385472585(6)&  218.81184385472594(10)&  218.8118438547251(9)\\
Spin down rate $\dot{\nu}$ (s$^{-2}$)&  $-4.083891(4)\times10^{-16}$&  $-4.083894(7)\times10^{-16}$&  $-4.08382(5)\times10^{-16}$\\
Proper motion in $\alpha$, $\mu_{\alpha}=\dot{\alpha}\cos \delta$ (mas~yr$^{-1}$)&  4.9177(11)&  4.9179(18)&  4.917(2)\\
Proper motion in $\delta$, $\mu_{\delta}=\dot{\delta}$ (mas~yr$^{-1}$)&
$-$3.917(2)&  $-$3.915(3)&  $-$3.913(4)\\
Parallax, $\varpi$ (mas)&  0.871(15)&  0.84(3)&  0.85(3)\\
Dispersion Measure\tablenotemark{b} (pc~cm$^{-3}$)&  15.9700&  15.9700& 15.9700\\
Orbital Period, $P_{\rm b}$ (day)&  67.82513826918(16)&  67.82513826935(19)&  67.82513826930(19)\\
Change rate of $P_{\rm b}$, $\dot{P}_{\rm b}$ ($10^{-12}$s~s$^{-1}$)&  0.23(12)&  0.41(16)&  0.44(17)\\
Eccentricity, $e$&  0.0000749394(3)&  0.0000749399(6)&  0.0000749402(6)\\
Time of periastron passage, $T_0$ (MJD)&  53761.03210(12)&  53761.0328(3)&  53761.0327(3)\\
Angle of periastron\tablenotemark{c}, $\omega$ (deg)&  176.1931(7)&  176.1967(15)&  176.1963(16)\\
Projected semi-major axis, $x$ (lt-s)&  32.34242253(6)&  32.34242188(14)&  32.34242188(14)\\
$\sin i$, where $i$ is the orbital inclination angle&  0.9708(16)&  0.951(4)&  0.951(4)\\
Companion Mass, $M_c$ ($M_{\odot}$)&  0.221(5)&  0.287(13)&  0.286(13)\\
Apparent change rate of $x$, $\dot{x}$ (lt-s~s$^{-1}$)&  0.00637(7)&  0.00640(10)&  0.00645(11)\\
Profile frequency dependency parameter, FD1 &  $-$0.00016271(19)&
$-$0.0001623(2)&  $-$0.00016(3)\\
Profile frequency dependency parameter, FD2 &  0.0001352(3)&  0.0001350(3)&  0.00014(3)\\
Profile frequency dependency parameter, FD3 &  $-$0.0000660(6)&
$-$0.0000668(6)&  $-$0.000067(17)\\
Profile frequency dependency parameter, FD4 &  0.0000145(4)&  0.0000153(4)&
0.000015(5)\\[4pt]
\textit{Fixed Parameters}&  &  &  \\%[1mm]
Solar system ephemeris&  DE421&  DE421&  DE421\\
Reference epoch for $\alpha$, $\delta$, and $\nu$ (MJD)&  53729&  53729&  53729\\
Solar wind electron density $n_{\rm 0}$ (cm~$^{-3}$) & 0 & 0 & 0 \\
Rate of periastron advance, $\dot{\omega}$ (deg/yr)\tablenotemark{d}&  0.00024&  0.00024&  0.00024\\
Position angle of ascending node, $\Omega$ (deg)\tablenotemark{e}&  88.43&  88.43&  88.43\\
Red Noise Amplitude ($\mu$s/${\rm yr}^{-1/2}$)&  --&  --&  0.025 \tablenotemark{f}\\
Red Noise Spectral Index, $\gamma_{\rm red} $&  --&  --&  $-$2.92\\[4pt]
\textit{Derived Parameters}&  &  &  \\%[1mm]
Intrinsic period derivative, $\dot{P}_{\rm Int}$(s~s$^{-1}$)\tablenotemark{*}&  $8.958(12)\times10^{-21}$&  $8.98(2)\times10^{-21}$&  $8.97(2)\times10^{-21}$\\
Dipole magnetic field, $B$ (G)\tablenotemark{*}&  $2.0475(14)\times10^{8}$&  $2.050(3)\times10^{8}$&  $2.049(3)\times10^{8}$\\
Characteristic age, $\tau_c$ (yr)\tablenotemark{*}&  $8.083(11)\times10^{9}$& $8.07(2)\times10^{9}$&  $8.07(2)\times10^{9}$\\
Pulsar mass, $M_{\rm PSR}$ ($M_{\odot}$)&  0.90(4)&  1.32(11)&  1.31(11)
\enddata
\tablenotetext{a}{We used a modified {\it DD} binary model \citep{dd86} that
allows us to assume a position angle of ascending node ($\Omega$) and fit for
the apparent change rate of the projected semi-major axis ($\dot{x}$) due to
proper motion. Numbers in parentheses indicate the 1 $\sigma$ uncertainties on the last
digit(s). Uncertainties on parameters are estimated by the {\it TEMPO} program
using information in the covariance matrix.}
\tablenotetext{b}{The averaged DM value; See Section 3.2 and Figure 2 for more discussion.}
\tablenotetext{c}{See Figure 2 of \citealt{sns+05} for definition.}
%\tablenotetext{d}{The model for estimating solar wind contribution in arrival time delay:
%$\Delta t_\odot=10^{14}n_0(\pi - \theta)/\sin\theta/r/f^2$~s, where $n_0$ is the solar wind
%electron density at 1 AU from the Sun, $\theta$ is the solar elongation of the
%pulsar, $r$ is the light travel time between the Sun and the observatory, and
%$f$ is the observing frequency in Hz.}
\tablenotetext{d}{The rate of periastron advance was not fitted but fixed to the GR value
because it is highly co-variant with the orbital period. }
\tablenotetext{e}{We optimized $\Omega$ using a grid search and held it fix to the value that
minimized $\chi^2$.}
\tablenotetext{f}{The value corresponds to $8.7\times10^{-15}$ in the dimensionless strain amplitude unit.}
\tablenotetext{*}{These parameters are corrected for Shklovskii effect and
Galactic differential accelerations.}


\end{deluxetable*}

%\clearpage 
