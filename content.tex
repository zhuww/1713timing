
\begin{abstract}
Pulsars are excellent testing grounds for fundamental physics. As precise
cosmic clocks, they have been used in many experiments, especially in testing
gravitational theories. We report 20-year timing of one of the most precise
pulsar---J1713+0747. We improve the measurements on the pulsar's parameters.
Most interestingly, we detected its orbital decay due to Galactic
differential acceleration, and use this measurement to greatly improve the generic
upper limit on the change rate of gravitational constant from pulsars. The new
limit shows that the change of gravitational constant has to be at least a
factor of 50 slower than the average expansion rate of the Universe. Such a
limit has important implications to alternative theories of gravity.
\end{abstract}


\section{Introduction}
\label{sec:intro}
\linenumbers
We present 20-year timing of the millisecond  PSR J1713+0747. Discovered in
1993 \citep{fwc93}, PSR J1713+0747 is one of the brightest pulsars timed by the
North American nano-Hertz Observatory for Gravitation Waves (NANOGrav;
\citealt{ndf+12, dfg+13}). It also has the smallest timing residual of all NANOGrav pulsars \citep{dfg+13}. Observed with the largest telescope in the world, the Arecibo Observatory, it is one of the most well-timed pulsar in the world.
Observations of this pulsar were reported previously in
\citealt{cfw94},\citealt{ vb03}, \citealt{sns+05}, and \citealt{hbo06}.

The high timing precision and the long base line allowed us to measure 
precisely 
the pulsar binary system's orbit, masses, distance, proper motion, 
and orientation on the sky (see Section \ref{sec:model} for
details). We also modeled the variation in the interstellar median
(Section \ref{sec:dmx}) and possibly small distortion in of its magnetosphere (Section
\ref{sec:FD}). 

MSP are very stable rotators thanks to their enormous angular momentum.
J1713+0747 is one of
the most stable among them, and it resides in a very wide binary orbit
with a $\sim0.3~M_{\odot}$ white dwarf (Table \ref{tab:par}). This make it the
perfect laboratory for observing the smallest cosmic variance, one such as the
change of gravitational constant as predicted in the Jordan-Fierz-Brans-Dicke
theory of gravity, ie.e the scalar tensor theory \citep{jor59,fie56,bd61}. 
As one of the last alternative theories of Gravity that is still not
ruled out, the scalar tensor theory describes a class of alternative theories of Gravity
that modifies Einstein's equation of dynamics by coupling mass with a scalar-tensor field.
This theory predicts, as the universe expands, the scalar field will
also change over time, thus lead to a changing gravitational constant. Despite
being cosmological time-scale phenomenon, it could still lead to observable 
changes in the orbits of binary systems like Luna Earth binary or pulsar
binaries in the matter of decades. The Luna Laser Ranging (LLR) experiment measured the
Luna-Earth distance to $10^{-11}$ precision for a 40 years time span, and thus put a
very good upper limit on the change rate of gravitational constant
\citep{hmb10}. Recently, thanks to the high-precision pulsar time array
projects, measurements of pulsar binary orbits have also been improved
greatly, reaching similar precision as the LLR experiment and start to provide
meaningful independent constrains on $\dot{G}/G$. PSR J1713+0747 is uniquely
useful in this particular test, because of its long orbital period. Unlike in
other pulsar binaries, where the orbital decay could be affected by
gravitational wave radiation (as predicted by Einstein's GR or by 
alternative theories), this system has very small gravitational radiations. 
Therefore, it is likely the best system for testing the constancy of
gravitational constant. We found that by combining the timing result of
J1713+0747 with that of other pulsars', we get a significantly improved upper limit
on $\dot{G}/G$ compare with previous timing-based limits (see Section
\ref{sec:Gdot} for detail).


\section{Observations}
Timing observation of PSR J1713+0747 started from 1993 at the Arecibo
Observatory. \citet{sns+05} reported the first 12 years timing result. After
that, there were about 5 years of gap time during which it was not observed by
Arecibo.  In 2005, NANOGrav started monitoring  J1713+0747 regularly as part
of  its pulsar timing array project. Since then the pulsar is observed monthly
in L band and S band by the Arecibo
observatory and in 800MHz and L band by the Green Bank Telescope (GBT).

We combine the pulse time of arrivals (TOA) used in \citealt{sns+05} and that
of the later NANOGrav observations. These timing observations were conducted 
using multiple generations of pulsar data acquiring systems. The early year
observations in \citet{sns+05} used
 the Princeton Mark III \citep{skn+92}, Mark IV
\citep{sst+00}, and the Arecibo Berkeley Pulsar Processor (ABPP) systems. 
In the beginning of NANOGrav observations, pulsar data are collected with the Astronomical Signal
Processor (ASP) and its Green Bank counterpart GASP \citep{dem07}. We
started using 
the Green Bank Ultimate Pulsar Processing Instrument (GUPPI; reference?) for GBT 
observations in 2010 and the Puertorican Ultimate Pulsar Processing Instrument
(PUPPI) for Arecibo observations in 2012. 
The date range, number of observation epochs and TOA, specification of the
system are listed in Table \ref{tab:obs}.

The latest system GUPPI and PUPPI have a much wider bandwidth (800~MHz), more than
a factor of 10 wider than their predecessor GASP and ASP (64~MHz). According to Radiometer 
equation, this bandwidth increase should lead to a factor of $\sim3$
smaller radiometer noise. Indeed some improvement in timing residual was
observed when upgrading from ASP to PUPPI, but less than predicted by the
radiometer equation. Perhaps a
significant part of the observed residuals are
not from pure radiometer noise, but from timing noise, jitter noise, or
gravitational wave signals (see Section \ref{sec:noise} for details).

\section{Timing model}
\label{sec:model}
A comprehensive timing model was described by \citet{sns+05}, this model includes the effect of pulsar rotation, astrometry, orbital motion, Shapiro delay and dispersion effect due to interstellar median.
We employed the same \citet{dd86} model they used to fit for the pulsar and
its orbital motions, we also used the DMX model to fit DM changes (see Section
\ref{sec:dmx} for detail), and we used an improved model to deal with profile
evolution in frequency (see Section \ref{sec:FD} for detail). 
We used the DE414(reference?) solar ephemeris instead of the DE405 used by
\citet{sns+05} for its improved precision on the masses of the Planets (?Is this
statement corret?).
In the pulsar model, we added an extra fitting parameter, the change rate of binary period $\dot{P}_{\rm b}$ to model a previously undetectable change in $P_{\rm b}$.This is described in Section \ref{sec:obdecay}.    
We can confirm that most of the pulsar-binary parameters that \citet{sns+05} got still works for the 20 years data. The new timing model parameters (Table \ref{tab:par}) differs from the \citet{sns+05} slightly but consistent with their reported uncertainties.

\subsection{Mass measurements}
\label{sec:mass}
The increased time span and advances in pulsar instruments has let to a much
more constraint timing model than the previous reported timing results. Most
notably the uncertainties on the pulsar's and the companion's masses have been
significantly reduced (Figure \ref{fig:mass}; Table \ref{tab:par}). The
companion's mass $M_{\rm c} = $ and the pulsar $M_{\rm psr}=$. They are in
good agreement with the previously measured values \cite{sns+05}.

Theoretical studies of binaries evolution in which MSP such as J1713+0747 was born  
indicate that there should be a correlation between the final orbital period
of the binary and the mass of the WD companion \citep{rpj+95, ts99a, prp02b}
. Such correlation is expected because the MSP must have been spun up by
accreting when its companion was a giant star. Therefore the WD
companion's progenitor must be big enough to fill Roche lobe. A wider
orbit leads to a larger Roche radius, and a more massive companion progenitor.
Indeed, evidence that supports this correlation has been found from few pulsar
binary systems \citep{vbb+01, ktr94}.  
J1713+0747's orbital period and companion mass also fits this theoretical
prediction very well.

Some early statistical analysis shows that pulsars 
might have a very narrow mass distribution 1.35$\pm0.04M_{\odot}$ \citep{tc99}. 
However, some massive ($\sim2M_{\odot}$) pulsars have been found in recent
years \citep{dpr+10, fbw+11, afw+13} that clearly out-lie this distribution.
It is believed that  J1713+0747 underwent an extended period of mass transfer,
but its mass still appears to be nominal when compared with the early year samples.


\subsection{DM variation}
\label{sec:dmx}
The DM of a pulsar characterizes how much interstellar median (ISM) lies
between the pulsar and us, and it changes as the ISM varies. For most pulsar
this change is too small to be detectable. However, for pulsars with residual
$\lesssim2\mu$s, the contribution from varying ISM start to contribute
significantly to the timing noise. 

Using the {\it DMX} model in {\it tempo}, we were able to fit for DM
variation. 
The model groups TOA from observations taken within ten days and fit for the
best DM for each group.   
It was applied to TOA from Mark4, ABPP, ASP, GAPS, GUPPI, and PUPPI data, but
not for Mark3. Since Mark3 had only one frequency band (L-band), a precise DM
fitting was not possible from its data.
J1713+0747 was observed in L- and S-band by the Arecibo Observatory. The
two bands observations almost always happen one after another on the same day.
While GBT observes J1713+0747 in 800~MHz and L band but the two band
observations often are separated by less than ten days.
Therefore, most DM values in the {\it DMX} model was determined by TOA from at
least two bands. Figure \ref{fig:dmx} shows the DM variation of J1713+0747 
measured using the {\it DMX} model.
Note that the sudden drop of DM around MJD-53000 was not real, it was caused
artificially by using a different pulse profile template for earlier-year Mark3/Mark4/ABPP TOA 
from the one used for the later
ASP/GASP/GUPPI/PUPPI TOA. The DM values inferred from these two different profile
templates are expected to be offset by a constant.
The pulsar shows a fractional DM changes of $\sim10^{-4}$ with variation of
long and short timescales and a significant dip around 2009 (as also seen from
\citealt{dfg+13}).



\subsection{Pulse profile evolution in frequency}
\label{sec:FD}
After removing the dispersion effect which corrects delays $\propto \nu^{-2}$
in the TOAs, small remaining frequency-dependent residuals were observed from
different instruments and telescopes (Figure \ref{fig:FD}).  
This phenomenon seems to be caused by the change of pulsar's pulse profile in
frequency, although the cause of such profile evolution is still unknown.
We speculate that different frequency pulsar radiation may be originated from
different parts of the star's magnetosphere, and that may have led to a
changing pulse profile over frequency.
In \citet{sns+05}, a jump was fitted between every two frequency channels in
order to remove such effect, however, with the a factor of $\sim$10 increase
of frequency channels from GUPPI, it would require many more jump parameters
to be added to the timing model.
In light by the observation that the frequency-dependent-residual seems to
trace a smooth function, we used a rank-4 polynomial of the logarithm of
frequency to fit and removed this effect from the final residual ({\it FD}
model by P. Demorest et~al in prep.;see Figure \ref{fig:FD}). 



\subsection{Timing Noise}
\label{sec:noise}
After remove the effect from changing interstellar median and a
frequency-dependent pulse profile, and fit for pulsar spin down and binary
effects, we got timing residuals $\lesssim 2\mu$s from J1713+0747 (Figure
\ref{fig:res}, top panel).
The residual looked smooth and flat, there were no obvious trend in time,
frequency, or orbital phase.
However, there seems to be still some timing noise (residual variance in time)
left in the upper band of the TOA (Figure \ref{fig:res}, bottom panel).

The absence of such timing noise in the average residual of high and low band is probably due to
our fitting of DM variation. 
This fitting process have a tendency to `absorb' part of the noises.
Therefore, we also expect that the weighted Root Mean Squares (WRMS) we measure from the
averaged residuals will be under-estimated. We can quantitatively
estimate this effect by injecting random Gaussian noise into the TOA and
measure the increase in post-fit WRMS. Figure \ref{fig:overfit} show
the post-fit residual as a function of injected noise for PUPPI L-band TOA. 
One can see that the increase is well fit by a linear function. 
Based on this experiment, we
can estimate the fraction of timing noise absorbed by DMX fitting, and use
this fraction to infer the original `amplitude' of the timing noises. 
Table \ref{tab:wrms} shows the WRMS measured directly from the
residuals and 
the inferred/recovered RMS based on noise injection simulations. One caveat of this
approach is that the simulated noise are white, while the actually timing
noise are likely red noises. 


The great advance in instrumentation should bring down the timing residual,
indeed, the best residual ($\sim100$~ns) come from PUPPI L-band observations.
\citet{sc12} studied the pulse arrival times from a long observation of
J1713+0747, and found that this pulsar's single pulse showed random jitter of
$\simeq40~\mu$s. Therefore, by averaging many pulses collected in the
$\sim20$~min NANOGrav observation, one expect $\sim 40$~ns of jitter noise. 
This could account for part of the observed PUPPI residuals.
Some of the extra residual could come from other source of timing noise or gravitational wave signals.
Ultimately, the timing solution and residuals presented in this paper could be used to put a stringent single-source upper limit on the stochastic gravitational wave background in the nano-Hertz region.

\begin{itemize}
\item T2 model versus DD? confirms that both model gives pretty much consistent results?
\end{itemize}

\section{Result}

\subsection{Intrinsic orbital decay}
\label{sec:obdecay}
Through timing modeling, we measured a significant orbital decay from
J1713+0747, $\dot{P}_{\rm b} = 0.77(18)\times10^{-15}$s~s$^{-1}$. 
%Such a precise measurement can be used to test Gravitational wave radiation theories of Einstein's and beyond.
It is expected that the motion of the binary relate to the observer will introduce extra orbital decay due to kinetic effects, i.e. radial acceleration effect \citep{dt91} and centrifugal acceleration effect (`Shklovskii' effect; \citealt{shk70}). Luckily, we have good measurement on the distance and proper motion of the binary system, which allow us to remove these effects 
and study the system's intrinsic orbital decay.
\begin{equation}
\dot{P}_{\rm b}^{\rm Acc} = \frac{A_{\rm G}}{c} P_{\rm b} =
-1.2\pm0.1\times10^{-13}~{\rm s~s^{-1}}
\end{equation}
where $A_{\rm G}$ is the line-of-sight acceleration of the pulsar binary,
this term is dominated by the difference in the Galactic accelerations of the
binary and our solar system, and is obtained using
Equation 5 in \citet{nt95} and Equation 17 in \citet{lwj+09}.
\begin{equation}
\dot{P}_{\rm b}^{\rm Shk} = (\nu_{\alpha}^2+\nu_{\delta}^2)\frac{d}{c}P_{\rm
b} = 6.8\pm0.3\times10^{-13}~{\rm s~s^{-1}}
\end{equation}
Therefore, the pulsar's intrinsic orbital decay is $\dot{P}_{\rm b}^{\rm Int}
= \dot{P}_{\rm b}^{\rm Obs} - \dot{P}_{\rm b}^{\rm Shk} - \dot{P}_{\rm b}^{\rm
Gal} = (0.7\pm2)\times10^{-13}$s~s$^{-1}$, and is consistent with being smaller than measurable.

There are two extra classical terms that could have played a role in
$\dot{P}_{\rm b}^{\rm Int}$, one $\dot{P}_{\rm b}^{\dot{M}}$ is caused by mass loss in the
binary system, the other $ \dot{P}_{\rm b}^{\rm T}$ is the contribution of tidal effect.
The pulsar could loss mass due to its radiation, the maximum mass loss rate
due to this effect orbit can be estimated from its rotational energy loss rate
$\dot{M_{\rm psr}}<\dot{E}/c^2$, and similar idea may apply to the white
dwarf. 
%\begin{equation}
%\dot{P}_{\rm b}^{\dot{m}} = 8\pi^2\frac{I_{\rm
%PSR}}{Mc^2}\frac{\dot{P}}{P^3}P_{\rm b} \sim 10^{-16},
%\end{equation}
%where $M=M_{\rm PSR} +M_{\rm WD}$ is the total mass of the system and
%$I\sim10^{45}$g~cm$^2$ is the angular momentum of inertia.
Consequently, the orbital change due to mass loss is $\dot{P}_{\rm
b}^{\dot{M}}\lesssim 1\times10^{-14}$s~s$^{-1}$ (\citealt{dt91}; Equation 9 and 10
of \citealt{fwe+12}), an order of magnitude smaller than the measured
uncertainties on $\dot{P}_{\rm b}^{\rm Int}$.
The tidal effect of this binary system is expected to be $\dot{P}_{\rm b}^{\rm
T}\ll1\times10^{-14}$s~s$^{-1}$ basing on the most extreme scenarios (the WD spins at
its break-up velocity and the tidal synchronizing time scale equals the
characteristic age of the pulsar; see Equation 11 in \citealt{fwe+12} and
references therein in).
Both of these classical terms are much smaller than the observed uncertainties
on $\dot{P}_{\rm b}^{Int}$.
%The tidal effect can be estimated to $\sim$ according to Equation 11 of
%\citealt{fwe+12} and references therein. 


Finally, due to the 68 day orbit, the binary's  gravitational wave
radiation is so weak that its orbital decay according to Einstein's
theory is expected to be 
$\dot{P}_{\rm b}^{\rm GR} = -7\times10^{-18}$s~s$^{-1}$ \citep{lk05}.
This is consistent with the observed intrinsic orbital decay, therefore,
Einstein's theory of gravitational wave radiation pass this particular test.


\subsection{Constraint on $\dot{G}$}
\label{sec:Gdot}
Some alternative theories of gravity, such as the scalar-tensor gravity, predict extra orbital decay than Einstein's theory. 
Consequently, measurements of pulsar binaries' `excess' orbital decay
$\dot{P}_{\rm b}^{\rm exc}=\dot{P}_{\rm b}^{\rm Int} - \dot{P}_{\rm
b}^{\dot{M}}  - \dot{P}_{\rm b}^{\rm T} - \dot{P}_{\rm b}^{\rm GR}$ have been
used to constrain them \citep{lwj+09, fwe+12}. 
The measured values of $\dot{P}_{\rm b}^{\rm exc}$ from the pulsars of
previous work were all consistent with being zero. Their upper limits were largely determined by the uncertainties of $\dot{P}_{\rm b}^{\rm Int}$.
Thanks to the high time precision of NANOGrav observations, we measure the
orbital period of J1713+0747 to extreme high precision, which enable us to 
get $\dot{P}_{\rm b}^{\rm exc}=(0.7\pm2)\times10^-13$s~s$^{-1}$ (Section
\ref{sec:obdecay}). 
This is the smallest fractional orbital decay rate observed in pulsar binaries
thanks to J1713+0747's exceptionally large $P_{\rm b}$, and can be used to put new limit on
the alternative theories of Gravity.

Extra orbital decay can be caused by the dipole gravitational radiation effect
introduced by some alternative gravity theories (\citealt{Will93, Will01, lwj+09, fwe+12} and reference therein). Such effect arise when the two bodies of binary are very different in terms of their self gravity, or in another word, their compactness.
This extra orbital decay is given by 
\begin{equation}
\dot{P}_{\rm b}^{\rm D} \simeq -4\pi\frac{T_{\odot}\mu}{P_{\rm b}}\kappa_D S^2,
\end{equation}
\citep{lwj+09}, where $T_{\odot}=G{\rm M_{\odot}}/c^3=4.9255$~${\rm \mu}$s, $\mu$ is the reduced mass $m_pm_c/M$ of the system , $\kappa_D $ is dipole
gravitational radiation "coupling constant", and $S$ is the difference
between the self-gravity "sensitivity" of the two bodies ($S = s_p - s_c$;
$s_p\sim0.1m_p/M_{\odot}$ according to \citealt{de92} ; and $s_c\ll s_p$).
In Einstein's GR $\kappa_D=0$ --- there is no self-gravity induced
dipole gravitational radiation, but it is often not the case in alternative
theories.

Another orbital decal term $\dot{P}_{\rm b}^{\dot{G}}$ is caused by a varying
local gravitational constant. In the framework of the scalar-tensor theories,
the scalar field that interacts with the mass has to change over time as the
Universe expands. This change will cause the locally measured gravitational
constant to vary. The changing gravitational constant will inevitably change
the orbit of a binary system:
\begin{equation}
\dot{P}_{\rm b}^{\dot{G}} = -2 \frac{\dot{G}}{G}
\left[1-\left(1+\frac{m_c}{2M}\right)s_p\right]P_{\rm b}
\end{equation}
\citep{dgt88,nor90}.

The orbital period of J1713+0747 is relatively long among the
high-time-precision pulsar binaries, this make its $\dot{P}_{\rm b}^{\rm D}$
very small. Conversely, $\dot{P}_{\rm b}^{\dot{G}}$ is larger when $P_{\rm b}$
is large. This makes J1713+0747 the best pulsar binary system for constraining
the change rate of gravitational constant $\dot{G}$. A combined limit on both
$\dot{G}$ and the dipole gravitational radiation constant $\kappa_D$ can be
done in the same fashion as in \citet{lwj+09}: solving $\dot{G}$ and $\kappa_D$
from this equation $\dot{P}_{\rm b}^{\rm exc} = \dot{P}_{\rm b}^{\rm D} +
\dot{P}_{\rm b}^{\dot{G}}$ (Equation 29 of \citealt{lwj+09}) of different
pulsars using Monte Carlo Markov Chain (MCMC) simulation. We applied this method to three pulsars: PSR J1012+5307, PSR
J1738+0333, and PSR J1713+0747 using timing parameter reported in
\citet{lwj+09}, \citet{fwe+12}, and this work.
The resulting confidence region of $\dot{G}$ and $\kappa_D$ is shown in Figure
\ref{fig:Gdot}.
We got, at 95\% confidence limit, $\dot{G}/G = -2\pm8\times10^{-13}$~yr$^{-1}$;
$\kappa_D=-0.3\pm2\time10^{-4}$. 
This constrain on $\dot{G}$ is more than a fact of two tighter than that from
previous pulsar-based constrain \citep{fwe+12}. 
Although slightly worse than
the best constrain of this type
($\dot{G}/G=0.7\pm7.6\times10^{-13}$~yr$^{-1}$) from the solar system LLR
experiment \citep{hmb10}, this new pulsar-timing  $\dot{G}$ limit is still an valuable independent test.  


%\section{Testing fundamental physics principles}
%\subsection{Conservation of momentum}
%\subsection{The Strong Equivalence Principle}

\section{Summary}
%over all result, most precise pulsar timing
In this paper, we present a comprehensive modeling of high precision timing observations of
PSR J1713+0747 that span 20 years. 
The pulse arrival times of this pulsar was well fit by our timing model with a
residual well below 2$\mu$s.
After correcting the effect of fitting, the `rescaled' WRMS in the post-fit residuals (Table \ref{tab:wrms}) provide a crude estimation of
the amount of noise still presence in the timing residual. The 20 years
overall residual is $\sim 250$~ns. The smallest WRMS $\sim100$~ns, as expected,
comes from PUPPI L-band observations.
This measurement has significant implications to estimating a single-source
upper limit of the stochastic gravitational wave background.


%Change rate of gravitational constant and limit on the dipole gravitational wave radiation.
In this work, we measure PSR J1713+0747's $P_{\rm b}$ to 11 significant
digits, and also detect its orbital decay $\dot{P}_{\rm b}$ due to Galactic differential accelerations.
When combined with other pulsars, the result of our $\dot{P}_{\rm b}$
measurement can help
significantly improve the current upper limit on the change rate of gravitational
 constant $\dot{G}$ from pulsar timing, making it almost as good as the best
constrain of this nature from the LLR experiment \citep{hmb10}.
The new best constrain based on pulsar timing is $\dot{G}/G
=2\pm8\times10^{-13}$~yr$^{-1}\ll0.02H_0$, where $H_0$ is the Hubble constant. . 
In another word, the change rate of gravitational constant has to be a factor of $>50$
slower than the Universe's average expansion rate.

Because $P_{\rm b}$ is a second order term in the timing 
model, its measured precision improves very fast $\delta
P_{\rm b} \propto T^{5/2}$, where $T$ is the total observational time span.
We can expect the precision of the above gravitation experiment from pulsar
timing improve very quickly and hopefully will surpass the precision of LLR
experiment in the future.
