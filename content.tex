
\begin{abstract}
Pulsars are excellent testing grounds for fundamental physics. As precise
cosmic clocks, they have been used in many experiments, especially in testing
gravitational theories. We report 20-year timing of one of the most precise
pulsar---J1713+0747. The results can be used to constrain alternative
gravitational theories and test the constancy of the gravitational constant.
\end{abstract}


\section{Introduction}
We present 20-year timing of the millisecond  PSR J1713+0747. Discovered in
1993 \citep{fwc93}, PSR J1713+0747 is one of the brightest pulsars timed by the
North American nano-Hertz Observatory for Gravitation Waves (NANOGrav;
\citealt{ndf+12, dfg+13}). It also has the smallest timing residual of all NANOGrav pulsars \citep{dfg+13}. Observed with the largest telescope in the world, the Arecibo Observatory, it is probably one of the most well-timed pulsar in the world.

The high timing precision and the long base line allowed us to measure precisely
the pulsar binary system's  masses, orbit, distance, proper motion, and even
the orientation of the orbit on the sky. More importantly, 
we measured the binary orbital period to very high precision, and detected a
very small orbital decay over the 20 years time span. This allow us to test
Einstein's theory of Gravity as well as some alternative theories.


The Jordan-Fierz-Brans-Dicke theory \citep{jor59,fie56,bd61} describes a class of alternative theories of Gravity. 
This class of theories modifies Einstein's equation of dynamics by coupling
mass with a scalar-tensor
field. Therefore, it is also referred to as the scalar-tensor gravity. 
This is one of the last alternative theories of Gravity that is still not
ruled out, although, fairly stringent constrain have been placed on the coupling
strength of the scalar field based on observations of binary systems
\citep{hmb10, lwj+09, fwe+12}.


\section{Observations}
Timing observation of PSR J1713+0747 started from 1993 at the Arecibo Observatory. \citet{sns+05} reported the first 12 years timing result. After that, J1713+0747 was monitored regularly by NANOGrav as part of its pulsar timing array project. In this project, J1713+0747 is observed monthly in L band and S band by the Arecibo Observatory and in 800MHz and L band by the Green Bank Observatory.

\section{Timing model}
\label{timing_model}
A comprehensive timing model was described by \citet{sns+05}, this model includes the effect of pulsar rotation, astrometry, orbital motion, Shapiro delay and dispersion effect due to interstellar median.
We employed the same \citet{dd86} model they used to fit for the pulsar and its orbital motions, we also used the DMX model to fit DM changes (see Section \ref{sec:dmx} for detail), and we used an improved model to deal with profile evolution in frequency (see Section \ref{sec:pfev} for detail). 
In the pulsar model, we added an extra fitting parameter, the change rate of binary period $\dot{P}_{\rm b}$ to model a previously undetectable change in $P_{\rm b}$.This is described in Section \ref{sec:obdecay}.    
We can confirm that most of the pulsar-binary parameters that \citet{sns+05} got still works for the 20 years data. The new timing model parameters (Table \ref{tab:par}) differs from the \citet{sns+05} slightly but consistent with their reported uncertainties.

\subsection{DM variation}
\label{sec:dmx}
The DM of a pulsar characterizes how much interstellar median (ISM) lies
between the pulsar and us, and it changes as the ISM varies. For most pulsar
this change is too small to be detectable. However, for pulsar with residual
$<\sim1\mu$s, the contribution from varying ISM start to contribute
significantly to the timing noise. 

Using the {\it DMX} model in {\it tempo}, we were able to fit for DM
variation. 
The model groups TOA from observations taken within ten days and fit for the
best DM for each group.   
It was applied to TOA from Mark4, ABPP, ASP, GAPS, GUPPI, and PUPPI data, but
not for Mark3, since only there were only one observable band (L-band) for
Mark3, so a precise DM fitting was not possible.
J1713+0747 was observed in L- and S-band by the Arecibo Observatory. The
two bands observations almost always happen one after another on the same day.
While GBT observe J1713+0747 in 800~MHz and L band but the two band
observations often are separated by only a few days.
Therefore, most DM values in the {\it DMX} model was determined by TOA from at
least two bands. Figure \ref{fig:dmx} shows the DM variation of J1713+0747 
measured using the {\it DMX} model.
The pulsar shows a fractional DM changes of $\sim10^{-4}$.



\subsection{Pulse profile evolution in frequency}
\label{sec:pfev}
After removing the dispersion effect which corrects delays $\propto \nu^{-2}$
in the TOAs, small remaining frequency-dependent residuals were observed from
different instruments and telescopes (Figure \ref{fig:FD}).  
This phenomenon seems to be caused by the change of pulsar's pulse profile in
frequency, although the cause of such profile evolution is still unknown.
In \citet{sns+05}, a jump was fitted between even frequency channels in order to remove such effect, however, with the a factor of $\sim$10 increase of frequency channels from GUPPI, it would require many more jump parameters to be added to the timing model.
In light by the observation that the frequency-dependent-residual seems to
trace a smooth function, we used a rank-4 polynomial of the logarithm of
frequency to fit and removed this effect from the final residual ({\it FD}
model by P. Demorest et~al in prep.;see Figure \ref{fig:FD}). 



\subsection{Timing Noise and DM variation}

\subsubsection{Covariance with DM changes}
The lower band residuals are covariant with the DM changes.


\begin{itemize}
\item T2 model versus DD? confirms that both model gives pretty much consistent results?
\end{itemize}

\section{Result}

\subsection{Intrinsic orbital decay}
\label{sec:obdecay}
Through timing modeling, we measured a significant orbital decay from J1713+0747, $\dot{P}_{\rm b} = ?$. 
%Such a precise measurement can be used to test Gravitational wave radiation theories of Einstein's and beyond.
It is expected that the motion of the binary relate to the observer will introduce extra orbital decay due to kinetic effects, i.e. radial acceleration effect \citep{dt91} and centrifugal acceleration effect (`Shklovskii' effect; \citealt{shk70}). Luckily, we have good measurement on the distance and proper motion of the binary system, which allow us to remove these effects 
and study the system's intrinsic orbital decay.
\begin{equation}
\dot{P}_{\rm b}^{\rm Acc} = \frac{A_{\rm G}}{c} P_{\rm b} =
-1.2\pm0.1\times10^{-13}~{\rm s~s^{-1}}
\end{equation}
where $A_{\rm G}$ is the line-of-sight acceleration of the pulsar binary,
this term is dominated by the difference in the Galactic accelerations of the
binary and our solar system, and is obtained using
Equation 5 in \citet{nt95} and Equation 17 in \citet{lwj+09}.
\begin{equation}
\dot{P}_{\rm b}^{\rm Shk} = (\nu_{\alpha}^2+\nu_{\delta}^2)\frac{d}{c}P_{\rm
b} = 6.8\pm0.3\times10^{-13}~{\rm s~s^{-1}}
\end{equation}
Therefore, the pulsar's intrinsic orbital decay is $\dot{P}_{\rm b}^{\rm Int}
= \dot{P}_{\rm b}^{\rm Obs} - \dot{P}_{\rm b}^{\rm Shk} - \dot{P}_{\rm b}^{\rm
Gal} = (0.7\pm2)\times10^-13$s~s$^{-1}$, and is consistent with being smaller than measurable.

There are two extra classical terms that could have played a role in
$\dot{P}_{\rm b}^{\rm Int}$, one $\dot{P}_{\rm b}^{\dot{M}}$ is caused by mass loss in the
binary system, the other $ \dot{P}_{\rm b}^{\rm T}$ is the contribution of tidal effect.
The pulsar could loss mass due to its radiation, the maximum mass loss rate
due to this effect orbit can be estimated from its rotational energy loss rate
$\dot{M_{\rm psr}}<\dot{E}/c^2$, and similar idea may apply to the white
dwarf. 
%\begin{equation}
%\dot{P}_{\rm b}^{\dot{m}} = 8\pi^2\frac{I_{\rm
%PSR}}{Mc^2}\frac{\dot{P}}{P^3}P_{\rm b} \sim 10^{-16},
%\end{equation}
%where $M=M_{\rm PSR} +M_{\rm WD}$ is the total mass of the system and
%$I\sim10^{45}$g~cm$^2$ is the angular momentum of inertia.
Consequently, the orbital change due to mass loss is $\dot{P}_{\rm
b}^{\dot{M}}\lesssim 1\times10^{-14}$s~s$^{-1}$ (\citealt{dt91}; Equation 9 and 10
of \citealt{fwe+12}), an order of magnitude smaller than the measured
uncertainties on $\dot{P}_{\rm b}^{\rm Int}$.
The tidal effect of this binary system is expected to be $\dot{P}_{\rm b}^{\rm
T}\ll1\times10^{-14}$s~s$^{-1}$ basing on the most extreme scenarios (the WD spins at
its break-up velocity and the tidal synchronizing time scale equals the
characteristic age of the pulsar; see Equation 11 in \citealt{fwe+12} and
references therein in).
Both of these classical terms are much smaller than the observed uncertainties
on $\dot{P}_{\rm b}^{Int}$.
%The tidal effect can be estimated to $\sim$ according to Equation 11 of
%\citealt{fwe+12} and references therein. 


Finally, due to the 68 day orbit, the binary's  gravitational wave
radiation is so weak that its orbital decay according to Einstein's
theory is expected to be 
$\dot{P}_{\rm b}^{\rm GR} = -7\times10^{-18}$s~s$^{-1}$ \citep{lk05}.
This is consistent with the observed intrinsic orbital decay, therefore,
Einstein's theory of gravitational wave radiation pass this particular test.


\section{Constraint on $\dot{G}$}
Measurements of intrinsic orbital decays of pulsar binaries have been used to
constrain alternative theories of gravitation such as the scalar-tensor gravity \citep{lwj+09, fwe+12}. 
These alternative theories predicts extra orbital decay than Einstein's
theory. Therefore, one can measures the upper-limit of this `excess' orbital decay
$\dot{P}_{\rm b}^{\rm exc}=\dot{P}_{\rm b}^{\rm Int} - \dot{P}_{\rm
b}^{\dot{M}}  - \dot{P}_{\rm b}^{\rm T} - \dot{P}_{\rm b}^{\rm GR}$ and use it to constrain the Gravitation theories.
The measured values of $\dot{P}_{\rm b}^{\rm exc}$ from previous work were all
consistent with being zero. Their upper limits were largely determined by the
uncertainties of $\dot{P}_{\rm b}^{\rm Int}$.
Thanks to the high time precision of NANOGrav observations, we measure the
orbital period of J1713+0747 to high precision, which enable us to 
constrain $\dot{P}_{\rm b}^{\rm exc}=(0.7\pm2)\times10^-13$s~s$^{-1}$. 
This is the smallest fractional orbital decay rate observed in pulsar binaries due to J1713+0747's long orbit period.
Therefore, the $\dot{P}_{\rm b}^{\rm exc}$ of J1713+0747 could be
combined with that of the other pulsar binaries to form the most stringent constraint on alternative theories of gravitation. 





%\section{Testing fundamental physics principles}
%\subsection{Conservation of momentum}
%\subsection{The Strong Equivalence Principle}

\section{Summary}
Compare generations of instruments

Radiometer equation; Limit to the timing precision; Pulse jitter noise.

Implication for Gravitational Wave upperlimit.

Change rate of gravitational constant and limit on the dipole gravitational
wave radiation.

Future prospect of Gravitation test by pulsar timing.
