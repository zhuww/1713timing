
\begin{abstract}
Pulsars are precise cosmic clocks and excellent testing grounds for
fundamental physics, especially for testing gravitational theories. 
We report 21-yr timing of one of the most precise
pulsars: PSR~J1713+0747. The pulsar's pulse times of arrival are well
modeled, with residuals having weighted root mean square of $\sim 92$~ns, by a
comprehensive pulsar binary model including the mass and 3D orbit of its
white dwarf companion and a noise model that incorporates short- and
long-timescale correlated noise such as jitter and red noise. 
The new dataset allows us to update and greatly improve previous measurements
of the system properties, including the masses of the neutron star
[1.31(11)~\Msun] and white dwarf [0.286(12)~\Msun] as well as their parallax 
distance [1.15(3)~kpc].  
We measured a change in the observed orbital period of PSR~J1713+0747, and
attribute this to the relative motion of the binary system and the Earth. 
The intrinsic change in orbital period, $\dot{P}^{\rm Int}_{\rm b}$, is $-0.20(17)$~ps~s$^{-1}$,
%$\dot{P}^{\rm Int}_{\rm b}=-0.20\pm0.17\times10^{-12}$s~s$^{-1}$, 
not distinguishable from zero.
This result, combined with the measured $\dot{P}^{\rm Int}_{\rm b}$ of other
pulsars, 
can place limits on potential changes in the gravitational constant $G$
as predicted in some alternative theories of gravitation. 
We found that $\dot{G}/G$ is consistent with zero
[$-0.6(1.1)\times10^{-12}$~yr$^{-1}$, 95\% confidence level] and changes at 
least a factor of $31$ (99.7\% confidence level) more slowly than the average expansion 
rate of the Universe. This is the current best $\dot{G}/G$ limit from pulsar binary
systems.
\end{abstract}


\section{Introduction}
\label{sec:intro}
\linenumbers
We present 21-yr timing of the millisecond pulsar (MSP) J1713+0747. This
pulsar was discovered in 1993 \citep{fwc93}. It is one of the brightest pulsars timed by the
North American Nanohertz Observatory for Gravitational Waves (NANOGrav;
\citealt{mcl13, dfg+13}), and has the smallest timing residual of all NANOGrav
pulsars \citep{dfg+13}.
The timing analysis reported in this paper will be incorporated to future pulsar timing array
projects.
%The timing analysis reported in this paper is pivotal to future pulsar timing array
%projects, such as the estimation of single-pulsar upper limit on the stochastic gravitational wave background.
Timing observations of this pulsar were reported previously in
\citealt{cfw94}, \citealt{lb01}, \citealt{vb03}, \citealt{sns+05}, \citealt{hbo06} and \citealt{ver09}.
%The high timing precision and the long base line allowed us to precisely 
%measure the pulsar binary's orbit, masses, distance, proper motion, 
%and orientation on the sky (see Section \ref{sec:model} for
%details). 

MSPs are very stable rotators due to their enormous
angular momentum. PSR~J1713+0747 is an MSP residing in a wide binary orbit with a white dwarf companion (Section \ref{sec:model}). 
The latest wide-bandwidth and high-resolution instruments allow us to model
and account for the variation due to dispersion in the interstellar medium (ISM)
(Section \ref{sec:dmx}) and small distortions of the pulsar's magnetosphere (Section \ref{sec:FD}). 
The pulse arrival times of the pulsar are well fit by a binary model with
a nearly circular orbit that is edge-on to our line of sight. The masses of
the binary can be inferred through the measurement of Shapiro delay \citep{sns+05}. 
The system's distance is well-measured through a timing parallax. We detect
a changing projected orbital semi-major axis due to the orbit's proper motion
on the sky. Through the rate of change of projected semi-major axis, we can infer 
the orientation of the orbit in the sky.
This is one of the few binaries in which the 3D-orientation of the
binary orbit can be completely solved. Using 21 yrs of data, we
refine the previously-published measurements of these orbital parameters.
We also find stringent constraint on the decay rate of the binary orbit due to the system's
motion (the Shklovskii effect and Galactic differential acceleration).
%These measurements also enable better tests of the theories of gravitation (see Section \ref{sec:res} for details).

The stability and long orbital period of the PSR~J1713+0747 binary make it the
perfect laboratory for observing the smallest cosmic variance, 
one such as the time variation of Newton's gravitational ``constant'' $G$. 
This interesting conjecture of $G$ varying on a cosmological timescale was first 
raised by \citet{dir37} based on his large-number hypothesis, and 
revisited later by gravitational theories involving extra dimensions
\citep{mar84,ww86a}.
In this paper, we seek to constrain the time variation of $G$ using this pulsar
binary in the framework of one of the last remaining alternative theories of gravitation, the scalar-tensor theory \citep{jor59,fie56,bd61}. 
This theory modifies Einstein's equation of gravitation by coupling mass with
a large-scale scalar-tensor field, and predicts that, as the universe expands,
the scalar field will also expand, causing the gravitational constant to vary  
on the cosmological timescale. 
PSR~J1713+0747 is likely the best pulsar binary for testing the constancy of
$G$ thanks to its high timing precision and long orbital period. Using timing
results reported in this paper, we found a stringent upper limit on 
$\dot{G}$ (see Section \ref{sec:Gdot} for details). 

The PSR~J1713+0747 binary is also an excellent laboratory for testing physical 
principles such as the strong equivalence principle (SEP) and the Lorentz invariance, because small violation of these principles could polarize the
binary orbit and result in potentially observable effects. We can set
upper limits on the violations of these principles by observing
low-eccentricity pulsar binary systems. Section \ref{sec:sep} presents the
constraint on the violation of SEP and Lorentz invariance from our 21 yrs of observation of PSR~J1713+0747.

\section{Observations}
Timing observations of PSR~J1713+0747 started in 1993 at the Arecibo
Observatory. \citet{sns+05} reported the results of the first 12 yrs
of timing observations. Since 2005 the pulsar is observed by NANOGrav approximately monthly
at 1400~MHz and 2300~MHz by the Arecibo
Observatory and at 800~MHz and 1400~MHz by the Robert C. Byrd Green Bank
Telescope (GBT).

The timing observations were conducted using multiple generations of pulsar
data backends (Table \ref{tab:obs}).
The earliest observations (1992--1994) used the Princeton Mark~III
\citep{skn+92}, which collects two polarization data with a filter bank of 32
spectral channels each 1.25~MHz wide. 
Observations between 1998 and 2004 used the Princeton Mark~IV
\citep{sst+00} instruments and the Arecibo-Berkeley Pulsar Processor
(ABPP; \citealt{bdz+97}) system  in parallel. 
The Mark~IV system collects 10-MHz passband data using 2-bit sampling. The
data were coherently dedispersed and folded at the pulse period offline.
The ABPP system samples voltages with 2-bit resolution and filters the passband 
into 32 spectral channels (1.75~MHz
per channel and 56~MHz in total for 1410-MHz band; 3.5~MHz per channel and 
112~MHz in total for 2380-MHz band), and applies coherent dedispersion to each
channel using 3-bit coefficients. 

From 2004 to 2011/12, pulsar data were collected with the Astronomical Signal
Processor (ASP; \citealt{dem07}) and its Green Bank counterpart GASP \citep{dem07}.
The (G)ASP systems record 8-bit sampled $\sim$64~MHz bandwidth data and apply real-time
coherent dedispersion and pulse period folding. The resulting data contain
2048-bin full-Stokes pulse profiles integrated over 1-3 minutes. 
When observing with ASP we used 20 channels each 4-MHz wide between 1440 and
1360~MHz, and 16 channels between 2318 and 2382~MHz. 
When observing with GASP we used 12 channels between 1386 and 1434~MHz, and 16
channels between 822 and 886~MHz.
The J1713+0747 ASP/GASP data were also reported in \citet{dfg+13}.

We started using 
the Green Bank Ultimate Pulsar Processing Instrument (GUPPI; \citealt{GUPPI}) for GBT 
observations in 2010 and its clone the ``Puerto-Rican Ultimate Pulsar Processing Instrument''
(PUPPI) for Arecibo observations in 2012. 
GUPPI and PUPPI use 8-bit sampling in real-time coherent dedispersion and
pulse period folding mode and produce 2048-bin full-Stokes
pulse profiles integrated over 10 second intervals.
When observing with 820-MHz receiver at the GBT, we use GUPPI to collect data from 62 spectral 
channels each 3.125~MHz wide, covering 724--918~MHz in frequency. With the L-band receiver, GUPPI
uses 58 spectral channels each 12.5~MHz wide, covering the 1150--1880~MHz band. 
When observing with the L-band receiver at Arecibo, we use PUPPI to collect data
from 1150--1765~MHz using 50 spectral channels each 12.5~MHz wide. When observing with the S band,
PUPPI takes data from 1770--1880 MHz and 2050--2405 MHz using 38 spectral
channels each 12.5~MHz wide.
The spectral bandwidth and resolution provided by GUPPI and PUPPI are crucial for resolving the pulse profile evolution in frequency described in Section \ref{sec:FD}.

The date span, number of observation epochs, specifications of the
systems are listed in Table \ref{tab:obs}. 

We combine the pulse time of arrivals (ToAs) used in \citealt{sns+05} and those
of the later observations, for a data span of 21 yrs, with a
noticeable gap between 1994 and 1998, during the Arecibo upgrade.
We used daily-averaged ToAs from \citealt{sns+05}, because the original 190~s integration ToAs are not accessible.
Data timestamps are derived from observatory masers and retrocorrected
to the Universal Coordinated Time (UTC) via GPS and then further
corrected to the TT(BIPM) timescale using the 2012 version BIPM clock corrections with extrapolations to 2013.
The ToAs are measured from the observational data through a series of
steps. First the data are folded, as they were being taken, into pulse
profiles using an ephemeris known to be good enough for predicting the
pulse period for the duration of the observation. The folded
profiles from different frequency channels and sub-integrations are
often summed together to improve the S/N of the profiles.  Orthogonal
polarizations are summed to produce a total-intensity profile.
The summed profiles are then compared with a well-measured standard
pulse profile from the appropriate frequency band. We employ
frequency-domain cross-correlation techniques \citep{tay92} to determine the phase of the pulse peak relative to the midpoint of the observation. The final ToA of a summed profile is then calculated by adding the mid-observation time and the product of pulse period and the measured peak phase.
The flux density of the pulsar in these observations can also be
measured by comparing the signal strength in the data with that of a
calibration observation taken right before or after the pulsar
observation in which a signal with known strength was injected. For
the post-upgrade Arecibo data and all the GBT data, the
flux density of the calibration signal is calibrated every month by
comparing it with an astronomical object of known and constant flux
density, in this case, the AGNs J1413+1509 and B1442+09.


The backends Mark~IV and ABPP were used in parallel for J1713+0747 
observations between 1998 and 2004. They collect data with different
bandwidths (Table \ref{tab:obs}). Mark~IV and ABPP ToAs were computed using different
profile templates.
To align the ABPP ToAs with Mark~IV, we
fitted a phase offset between the 1410-MHz ToAs of the two instruments, and found
that ABPP 1410-MHz ToAs trail those of Mark~IV by $\simeq0.46791$ 
%$0.46791205$
in pulse phase.
However, due to the difference in bandwidth and center frequencies, the S-band
ABPP ToAs trail Mark~IV slightly more than aforementioned, and the exact
offset value depends on our model of the profile evolution in frequency (FD
model; see Section \ref{sec:FD} for details). Therefore, in
addition to fixing the aforementioned phase offset, we also 
fit the S-band ABPP ToAs with an extra time offset as an unknown parameter.
%However, the same phase offset does not align the S band ToAs of Mark~IV and
%ABPP. 

During the transition from Mark~IV/ABPP to the ASP backend, Mark~IV
was kept running in parallel with ASP for several epochs. 
The overlapping Mark~IV data were not published in previous papers. 
We extracted ToAs from these overlapping Mark~IV data (labeled as Mark~IV-O in Table \ref{tab:obs})
using the same template used for the published Mark~IV ToAs,
and determined that the 1410-MHz ToAs of Mark~IV trail that of ASP by
2.33(10)~$\mu$s. Consequently, the
1410-MHz ToAs from Mark~IV, ABPP, (G)ASP, and (G)PUPPI form a connected time
series spanning 16~yrs within which no unknown time offset has to be fitted. 
Unfortunately, there are no
over-lapping observations connecting these 16-yr data with the earliest  Mark~III ToAs. 
Time offsets have to be
fitted for the 1410-MHz Mark~III ToAs as unknown parameters in the timing
model. The resulting timing solution is still
clearly phase connected because the uncertainties on these time offsets are
much
smaller than one spin period of the pulsar. However, these time offsets will
slightly inhibit our ability to detect small variations in timing-derived
parameters on time scale longer than 16 yrs.

The backend instruments introduce varying degrees of computational
and electronic delays into the measured ToAs. 
As a result, the ToAs from different backends are different by small time
offsets.
In order to measure these offsets, we allow data to be taken 
with both backends in parallel during a period of time, usually about few years.
As a result, the offset between GASP and GUPPI, and the offset between ASP and
PUPPI are well measured from these parallel observations using all NANOGrav
pulsars. These offsets are fixed when modeling the ToAs. 




\section{Timing model}
\label{sec:model}
%We model the pulsar's timing behavior with the comprehensive model described by \citet{sns+05},
%which included the effects of pulsar spin, astrometry, orbital motion,
%Shapiro delay and dispersion due to the ISM.
We employ the pulsar timing packages {\sc tempo}
\footnote{\url{http://tempo.sourceforge.net}} and {\sc tempo2} \citep{hem06} to model the ToAs. 
The rotation of the pulsar is modeled with polynomials of spin frequency 
$\nu$ and $\dot{\nu}$ (Table \ref{tab:par1} and \ref{tab:par2}) in order to account
for the pulsar's spin and spin down.
The pulsar's position ($\alpha$, $\delta$) and proper motion ($\nu_\alpha$, $
\nu_\delta$) on the sky and its parallax $\pi$ are also measured through timing modeling. 
The distance of the pulsar can be inferred accurately from our parallax
measurement $D_{\rm PSR}=$1.15(3)~kpc. This distance is consistent with and
more accurate than the VLBA parallax distance of 1.05(6)~kpc \citep{cbv+09}.
We employed the \citet{dd86} (DD) model of binary motion to fit for the binary parameters, 
including the masses of
the neutron star ($M_{\rm PSR}$) and the white dwarf ($M_{\rm c}$) measured
through Shapiro delay (Section \ref{sec:mass}),
the orbital period $P_{\rm b}$, angle of periastron $\omega$, time of
periastron passage $T_0$, projected semi-major axis $x$ and its change rate
$\dot{x}$ due to proper motion of the orbit. 
We observed an apparent $\dot{x}$ because the binary orbit moved perpendicular
to our line of sight and caused the projection angle to change over time. This 
allowed us to determine the orientation of the orbit in the sky when combined
with the system's proper motion.
The orientation of the orbit in the sky is modeled by the
parameter $\Omega$, which represents the position angle of the ascending node.
In {\sc tempo}, we grid search for the best $\Omega$ and then hold it fixed when fitting other
parameters (Table \ref{tab:par1}).
In the T2 model of {\sc tempo2}, $\Omega$ is explicitly modeled and
fitted, while $\dot{x}$ is implicitly modeled and not fitted as a parameter
(Table \ref{tab:par2}).  
Compared with previous timing efforts, we detected, for the first time, a
significant decay rate of binary period $\dot{P}_{\rm b}$ due to relativistic
effects. This is described in Section \ref{sec:obdecay}.    
%The T2 model of binary motion in {\sc tempo2} \citep{hem06} also can be used for modeling this pulsar. We tested the T2 model and found that its best-fit binary parameters are physically consistent with those we found using DD model.
The DMX model was used to fit dispersion measure (DM) variations caused by
changes in the ISM along the line of sight (see Section \ref{sec:dmx} for details). The FD model was used to model profile
evolution in frequency (see Section \ref{sec:FD} for details). 

In order to account for unknown systematics in ToAs from different
instruments, and epoch-correlated noise such as pulse jitter noise from pulsar
magnetospheric activities, we employed a general noise model that parameterizes both uncorrelated and
correlated noise. The noise modeling is discussed in Section \ref{sec:noise} and the
noise model parameters are listed in Table \ref{tab:wrms}. 
%we use the EQUAD command in {\sc tempo} to increase the errors on
%the ToAs by adding an extra amount of error to them in quadrature, such that
%the post-fit reduced $\chi^2$ is close to unity for the ToAs of every instrument. The amount
%of extra error added for different instruments are listed in Table
%\ref{tab:equad}.

%We use the DE421 solar ephemeris instead of the DE405 used by
%\citet{sns+05} for its improved precision on the masses of the Solar system
%planets, despite that using DE405 gave us marginally better $\chi^2$
%($\Delta\chi^2\sim20$ for 16750 degrees of freedom).
We used the JPL DE421 solar system ephemeris \citep{fwb09} to remove pulse
time-of-flight variation within the solar system. This ephemeris is oriented
to the International Celestial Reference Frame (ICRF) and thus our astrometric
results are also given in the ICRF frame. We note parenthetically that a
previous generation ephemeris, DE405, gave nearly identical but marginally
better timing fits ($\Delta\chi^2\sim6$ for 14528 degrees of freedom in both
{\sc tempo} and {\sc tempo2}).

The timing parameters and uncertainties are marginalized using a general least
square approach used by the {\sc tempo} software package. 
We confirm that majority of the pulsar-binary parameters reported by
\citet{sns+05} also successfully describe the 21-yr data set. Using a new
noise modeling technique, we detected 
a significant red noise signal that could be the same ``timing noise'' detected by
\citet{sns+05}.
We also detected for the first time an apparent decay of the binary orbit due to
relativistic effects.
The new timing model parameters (Table \ref{tab:par1} and \ref{tab:par2}) changed slightly from
those in \citet{sns+05} but are consistent with their reported uncertainties.

%The parameters uncertainties are estimated from a Markov Chain Monte Carlo
%\footnote{{\bfref We used a Metropolis algorithm assuming the goodness of fit of
%the timing model is proportional to $e^{-\chi^2/2}$. The procedure is similar
%to that in the {\sc tempo2} MCMC plugin \citep{dpr+10}. The code for running this
%MCMC can be found here: \url{https://github.com/zhuww/tempomcmc}}}  
%(MCMC; \citealt{NR3}) simulation to determine the parameter uncertainties (Table \ref{tab:par}).


%\begin{itemize}
%\item T2 model versus DD? confirms that both model gives pretty much consistent results?
%\end{itemize}

\subsection{Noise model}
\label{sec:noise}
The noise model used in this analysis is a parameterized model that is a function of several unknown quantities describing both correlated and uncorrelated noise sources. Uncorrelated noise is 
independent from one ToA to another, while the correlated noise is not. 
For instance, the template matching error and the radiometer noise are
uncorrelated in time, but the pulse jitter noise \citep{sc10}, which affects all the ToAs 
in the same epoch simultaneously, is correlated in time.
There is also time-correlated noise, such as red timing noise that is
correlated from epoch to epoch. Among the various types of noise only the
template matching error $\sigma$ can be estimated when we compute the ToAs.
Other source of noise must be modeled separately. The solution to this problem has been discussed extensively in \citet{vl13, ell13, vv14a, vv14, abb+14} and J. A. Ellis (PhD thesis 2014). In this section, we follow the noise model used in \citet{abb+14} closely but will make a few modifications.

We begin by forming a set of residuals via the standard weighted least squares fitting routine. An $N_{\rm toa}$ length vector of  residuals can be modeled mathematically as the sum of several deterministic and stochastic sources as follows
\be
\delta\mathbf{t} = M\boldsymbol{\epsilon} + F\mathbf{a} + U\mathbf{j} +\mathbf{n}.
\ee
The first term on the right hand side ($M\boldsymbol{\epsilon}$) describes small deterministic trends
left over in the residuals because one cannot perfectly remove the
deterministic timing model when noise is present. Here $M$ is the timing model
design matrix and $\boldsymbol{\epsilon}$ is a vector of small timing model
parameter offsets. Next term $F\mathbf{a}$ models the red noise via a Fourier decomposition
(this need not be the case as this basis was chosen to increase computational
efficiency as we shall see later) where $F$ is the so-called Fourier design
matrix that has columns of alternating sine and cosine functions for
frequencies in the range $[1/T,n_{\rm mode}/T]$ with $T$ being 
the total observation time span, $\Delta f=1/T$, and
$n_{\rm mode}$ being the number of frequencies to include in the sum.
Furthermore, $\mathbf{a}$ are the amplitudes of the Fourier basis functions
defined above \citep[see][for more details]{lah+13,abb+14}. 
The term $U\mathbf{j}$ describes short timescale correlated noise. This term could be due to pulse
phase jitter but could also have other components not due to jitter. This term
characterized noise that is completely correlated for all ToAs in a given
epoch but completely \emph{uncorrelated} between epochs. The matrix $U$ is an
$N_{\rm ToA}\times N_{\rm epoch}$ matrix that maps ToAs to a given epoch and
$\mathbf{j}$ is the amplitude of the epoch-to-epoch fluctuations. Finally the
last term $\mathbf{n}$ describes a Gaussian white noise process which
characterizes epoch, time, and frequency independent random noise left in the data. 

Since the white noise is modeled as Gaussian, the likelihood function for the noise is given by
\be
p(\mathbf{n}|\boldsymbol{\epsilon}, \mathbf{a}, \mathbf{j}) = \frac{\exp\left( -\frac{1}{2}\mathbf{n}^TN^{-1}\mathbf{n} \right)}{\sqrt{\det(2\pi N)}},
\ee
where 
\be
N_{ij,k}=E_{k}^2(W_{ij}+Q_{k}^2\delta_{ij}),
\ee
is an $N_{\rm ToA}\times N_{\rm ToA}$ matrix with $E_{k}$ and $Q_{k}$
corresponding to {\sc tempo} and {\sc tempo2}'s ``EFAC" and ``EQUAD" parameters for each observing backend, respectively, $W={\rm diag}\{\sigma_i^2\}$ is a diagonal matrix of ToA uncertainties, and $\delta_{ij}$ is the Kronecker delta function. The notation is such that the matrix elements $(i,j)$ apply to those ToAs corresponding to the backend observing system labeled by $k$. We can now write the likelihood function of the residuals as
\be
p(\delta\mathbf{t}|\boldsymbol{\epsilon}, \mathbf{a}, \mathbf{j},\boldsymbol{\phi}) = \frac{\exp\left( -\frac{1}{2}\mathbf{r}^TN^{-1}\mathbf{r} \right)}{\sqrt{\det(2\pi N)}},
\ee
where $\boldsymbol{\phi}$ describes the EFAC and EQUAD parameters and 
\be
\mathbf{r} = \delta\mathbf{t}-M\boldsymbol{\epsilon}-F\mathbf{a}-U\mathbf{j}
\ee
We now wish to impose prior distributions on our short timescale correlated noise and red noise. Both can be modeled as Gaussian processes by imposing the following priors
\begin{align}
p(\mathbf{j}|J_{k}) &= \frac{\exp\left( -\frac{1}{2}\mathbf{j}^{T}\msJ^{-1}\mathbf{j} \right)}{\sqrt{\det(2\pi\msJ)}}\\
p(\mathbf{a}|\rho_{n}) &= \frac{\exp\left( -\frac{1}{2}\mathbf{a}^{T}\varphi^{-1}\mathbf{a} \right)}{\sqrt{\det(2\pi\varphi)}},
\end{align}
where $\msJ_{ij,k} = J_{k}^{2}\delta_{ij}$ is an $N_{\rm epoch}\times N_{\rm
epoch}$ matrix with diagonal elements, $J_{k}^{2}$ describes the variance of
the jitter-like correlated noise for each observing backend. It is also
referred to as the ECORR parameter in {\sc tempo} and {\sc tempo2}. Furthermore $\varphi_{ij} = {\rm diag}\{10^{\rho_{n}}\}$ is an $2n_{\rm mode}\times 2n_{\rm mode}$ matrix describing the variance of the red noise Fourier coefficients at each frequency. In this framework, the coefficients of the $\varphi$-matrix are related to the power spectral density evaluated at a given frequency via the Wiener-Khinchin theorem. In principle we could use the power spectrum coefficients, $10^{\rho_{n}}$, themselves as free parameters but in this work we parameterize them via a power law
\be
\varphi_{n} \equiv 10^{\rho_{n}} = \frac{1}{T} A_{\rm red}^2\left( \frac{f}{f_{\rm yr}} \right)^{\gamma_{\rm red}}
\ee
where $T$ is the total observation time, $A_{\rm red}$ is the amplitude of the
red noise in $\mu$s~${\rm yr}^{1/2}$, $\gamma_{\rm red}$ is the spectral index of the red noise, $f_{\rm
yr}$ is the reference frequency of 1~${\rm yr}^{-1}$, and $f_{n}$ is the $n$th
Fourier frequency assuming Nyquist sampling. We see that the prior
distributions on jitter-like correlated noise and red noise are themselves
parameterized by some combination of \emph{hyper-parameters}. 
We can write down the posterior distribution for the residuals
\be
p(\boldsymbol{\epsilon}, \mathbf{a}, \mathbf{j},\boldsymbol{\phi}|\delta\mathbf{t}) \propto
p(\delta\mathbf{t}|\boldsymbol{\epsilon}, \mathbf{a}, \mathbf{j},\boldsymbol{\phi})p(\mathbf{j}|J_{k})p(\mathbf{a}|\rho_{n}),
\ee

For the purposes of estimating the underlying noise characteristics of our
data set, the parameters $\mathbf{\epsilon}$, $\mathbf{j}$, $\mathbf{a}$ are
nuisance parameters that we wish to marginalize over. This can be done in a
sequential fashion as was presented in \cite{abb+14} but here we take a
different approach. Notice that all timing parameters are ``linear'' and can be described with Gaussian prior distributions\footnote{We use uniform priors on the timing model parameter offsets, $\mathbf{\epsilon}$ but this is the same as a Gaussian prior with infinite variance. Technically this prior is not normalizable but since we are interested in parameter estimation and not Bayesian model selection here, this non-normalizable prior is not a problem.}. We can thus define a combined operator matrix and amplitude vector
\be
T = \bb M  & F  & U \eb, \quad
\mathbf{b} = \bb \boldsymbol{\epsilon} \\ \mathbf{a} \\ \mathbf{j} \eb
\ee
with prior distribution
\be
p(\mathbf{b}|\boldsymbol{\theta}) = \frac{\exp\left( -\frac{1}{2}\mathbf{b}^{T}B^{-1}\mathbf{b} \right)}{\sqrt{\det(2\pi B)}}
\ee
and covariance matrix is defined in terms of the following block matrix
\be
B = \bb \infty & 0 & 0 \\ 0 & \varphi & 0 \\ 0 & 0 & \msJ \eb,
\ee
where $\infty$ is a diagonal matrix of infinities to describe a uniform prior on $\boldsymbol{\epsilon}$. The marginalized posterior distribution is then
\be
\begin{split}
p(\boldsymbol{\phi}|\delta\mathbf{t}) & \propto \int_{-\infty}^{\infty}d\mathbf{b}\, p(\boldsymbol{\epsilon}, \mathbf{a}, \mathbf{j},\boldsymbol{\phi}|\delta\mathbf{t}) \\
&= \frac{\exp\left[-\frac{1}{2}(\delta\mathbf{t}^{T}N^{-1}\delta\mathbf{t} - \mathbf{d}^{T}\Sigma^{-1}\mathbf{d})  \right]}
{\sqrt{(2\pi)^{N_{\rm ToA}-{\rm dim\,}\mathbf{b}}\det(N)\det(B) \det(\Sigma)}},
\end{split}
\ee
where
\begin{align}
\mathbf{d} &= T^{T}N^{-1}\delta\mathbf{t} \\
\Sigma &= (B^{-1} + T^{T}N^{-1}T).
\end{align}
Furthermore, the maximum likelihood values of $\mathbf{b}$ and their uncertainties can be found by
\begin{align}
\hat{\mathbf{b}} &= \Sigma^{-1}\mathbf{d} \\
{\rm cov}(\mathbf{b}) &= \Sigma^{-1}.
\end{align}
This scheme has the advantage of being computationally efficient in that it
bypasses $O(N_{\rm ToA}^{3})$ matrix operations via rank reduced matrices
\citep{vv14} resulting in a likelihood evaluation that instead scales as
%$N_{\rm ToA}^{2}$; a factor of $\sim 2\times 10^{4}$ for this dataset.
$O(N_{\rm par}^{3})$, where $N_{\rm par}$ is the sum of the number of timing
parameters, red noise sample frequencies, and observing epochs. For this
dataset the computational speed up is a factor of $\sim10^{3}$.


For a given set of hyper-parameters, this allows us to determine the maximum
likelihood timing model parameters via the equivalent of a generalized least
squares (GLS) fit and the maximum likelihood red noise realization present in the data. 
We can also evaluate the posterior of the hyper-parameters $\boldsymbol{\phi}$
and thus find the maximum likelihood noise parameters including the EFAC, EQUAD,
ECORR and red noise amplitude $A_{\rm red}$ and spectral index $\gamma_{\rm
red}$ (Table \ref{tab:wrms}).
The posterior of the noise parameters are sampled using a Markov Chain Monte
Carlo process in which we sample some parameters in $\log_{10}$ space and
limit them to $\log_{10}J_{k}\in[-8.5, -4]$, with $J_{k}$ in units of seconds,
$\log_{10}A_{\rm red}\in[-7.5,1.5]$ where $A_{\rm red}$ is in units of
$\mu$s/${\rm yr}^{-1/2}$, and $\gamma_{\rm red}\in[-7,0]$. 


In our noise model we include EFAC, EQUAD, and ECORR parameters
for data collected by different backend and receiver systems. However, we do not model 
EFAC and ECORR of the Mark~III, Mark~IV, and ABPP data because there were not
enough ToAs in the legacy dataset to constrain both EFAC and EQUAD. Furthermore,
there was only one ToA per epoch so we cannot constrain the epoch-correlated
noise modeled by ECORR. Instead, we set EFAC values to 1
and ECORR to 0 for these data sets, and use only EQUAD to model the white noise in these epochs.



\citet{sc12} studied the pulse arrival times from a single long exposure of
PSR~J1713+0747, and found that this pulsar's single pulses showed random jitter of
$\simeq26~\mu$s. 
A similar result of $\simeq27~\mu$s was found by \citet{dlc+14} from a more recent study using a 24-h continuous observation of PSR~J1713+0747 conducted with major telescopes around the globe.
Therefore, by averaging many pulses collected in the typical
$\sim20$~min NANOGrav observation, one expects $\sim27\mu{\rm s}/\sqrt{1200\nu}=51$~ns of jitter noise. 
Table \ref{tab:par1} and \ref{tab:par2} show the best-fit timing parameters before and
after we applied our noise model to the data. It is clear that the jitter
noise significantly impacted the arrival time of the pulses, such that 
some timing parameters changed significantly after including the jitter model.
The optimal jitter parameters (ECORR, as shown in Table \ref{tab:wrms}) from
our noise modeling are mostly consistent with the prediction from
\citet{sc12}, with some of them being higher. This could be due to the
covariance between the jitter parameters and the EQUAD parameters.

In Figure \ref{fig:res} we show the red noise realization based on our best
noise model (Table \ref{tab:wrms}) and compare it to the post-fit residuals of
a {\sc tempo} general least squares (GLS) fit. The bottom
panels of Figure \ref{fig:res} show the one- and two-dimensional posterior probability plots
of the red noise. This noise model describes the data very well as we can see in Figure \ref{fig:detres} in which the maximum likelihood realizations of red and jitter noise are subtracted out. We see from the figure that both the high and low frequency residuals (with red and jitter noise realizations subtracted) are white (described by our EFAC and EQUAD parameters) and the weighted residuals follow a zero-mean unit-variance Gaussian distribution.

%Unlike in our reduced rank
%approach, in the {\sc tempo} 
%GLS fit the big covariance matrix $N$ of the entire dataset
%is inverted and the correlated red noise terms in the covariance matrix are 
%computed with an analytical expression derive from the power law spectrum.
%The resulting {\sc tempo } GLS residual was well fit by our red noise realization.
%In fact, if we subtract these two, the difference residuals, as shown in
%Figure \ref{fig:detres}, are well consistent
%with white noise modeled by our EFAC and EQUAD parameters.

%In Figure \ref{fig:res} we show the red noise realization based on our best
%noise model (Table \ref{tab:wrms}) and compare it to the post-fit residual of
%a {\sc tempo} general least square (GLS) fitting. Unlike in our reduced rank
%approach, in the {\sc tempo} 
%GLS fit the big covariance matrix $N$ of the entire dataset
%is inverted and the correlated red noise terms in the covariance matrix are 
%computed with an analytical expression derive from the power law spectrum.
%The resulting {\sc tempo } GLS residual was well fit by our red noise realization.
%In fact, if we subtract these two, the difference residuals, as shown in
%Figure \ref{fig:detres}, are well consistent
%with white noise modeled by our EFAC and EQUAD parameters.

The red noise signal seen in the data (i.e. Figure \ref{fig:res}) does not
resemble a low-frequency sinusoid or cubic
but instead it seems to be modeling a feature that is
localized to the time period of 1999 through 2005. 
A similar feature was also observed by \citet{sns+05}, and they modeled it
using higher order (up to the 8th order) frequency polynomials.
Since this is when
dual-frequency observations begin, it is possible that part of the red noise
signal contains contributions from unmodeled DM variations in data before 1999, 
thus making physical interpretations of red noise difficult because of the lack of multi-frequency data at early epochs.
A second possibility is that there were instrumental drifts in either the
Mark~IV or the APBB. This possibility can only be tested by examining data of
this pulsar taken by other telescopes and instruments from the same period of
time. 

%Fitting for DM variations on short timescales is necessitated by the sharp
%variations seen in DM (Figure \ref{fig:dmx}). However, this has the potential
%effect of overfitting the model and artificially reducing the spread in the
%residuals.
%This is especially significant for the lower band residuals, because
%the DM fitting affects the lower frequency ToAs more than the
%higher frequency ToAs, and is more likely to flatten their timing residuals. 
%As a result we expect the residuals' WRMS to be 
%underestimated. This effect can be crudely quantified
%by injecting random Gaussian noise to the ToAs and
%measuring the response in the post-fit residuals. 
%We perform this simulation one system at a time. 
%We added Gaussian noise to the ToAs of each observation epoch of the giving system.
%The noise are random between different epochs, but they are the
%same for all the ToAs of different frequencies in any given epoch. 
%Figure \ref{fig:overfit} shows
%the WRMS of the post-fit residuals as a function of injected
%noise level for the PUPPI 1400-MHz ToAs. 
%One can see that the RMS residual increases almost linearly as a function of the
%injected noise, but only a fraction of the injected noise is recovered in the timing 
%residual. Based on such simulations, we
%can estimate the level of noise ``absorbed'' by the timing models, and infer 
%the ``corrected amplitude'' of the timing residuals. 
%Table \ref{tab:wrms} shows the WRMS measured directly from the residuals 
%and the ``corrected RMS'' based on noise injection simulations. 
%These ``corrected RMSs'' serve to demonstrate the effect of ``overfitting''.
%They are not the proper measurements of noise level in the data. There are some 
%caveats in our approach, firstly, we only simulate noise between different
%epochs, not noise between different frequencies of the same epoch; secondly, 
%the injected noise is Gaussian white noise, while the actual 
%noise may contain some red and non-Gaussian noise. 






\subsection{Mass measurements}
\label{sec:mass}
The timing model of PSR~J1713+0747 has been significantly improved by the 21-yr timing effort.
Most notably, the pulsar and the companion masses have been more precisely
constrained (Table \ref{tab:par1} and \ref{tab:par2}) through Shapiro delay measurements. The
companion's mass $M_{\rm c} = 0.286\pm0.012$~\Msun\, and the pulsar $M_{\rm
PSR}=1.31\pm0.11$~\Msun\, are in good agreement with the previously measured
values \citep{sns+05}.
We note that the derived pulsar mass tends to be significantly underestimated
($\Delta M_{\rm PSR} \sim 0.3 M_{\odot}$) if no jitter parameters were included in our noise model (Table
\ref{tab:par1} and \ref{tab:par2}), suggesting that
correlated noise could significantly impact the result of high
precision timing analysis.


The pulsar's mass is comparable with the distribution of pulsar masses
in other neutron star-white dwarf systems, and in good
agreement with the distribution of pulsar masses found in recycled binaries
\citep{opns12,kkdt13}. The precise measurement of neutron star masses like this
may eventually help us understand the properties of matter in extreme 
density \citep{lat12}.

In the standard picture of binary evolution, an MSP with a low-mass white dwarf companion must have been spun up through accretion when the white dwarf was a giant star filling its Roche lobe. 
This should lead to a strong correlation between the binary period and the mass of the white dwarf companion \citep{rpj+95, ts99a, prp02b}. 
Indeed, this picture has been supported by the measurements of several pulsar
binary systems \citep[e.g.,][]{vbb+01, ktr94, th14}.  
The orbital period and companion mass of PSR~J1713+0747 fits
this correlation very well, thus supporting the standard MSP evolution theory. %Notably, not all MSP companions follow the same orbital period -- companion
%mass relation, for instance, J1903+0327 has an orbit of 95~days but a
%companion of $\sim 1M_{\odot}$ \citep{fbw+11}. Such peculiar systems requires
%more than the simple scenario considered above to explain.

%Some early statistical analysis shows that pulsars 
%might have a very narrow mass distribution 1.35$\pm0.04M_{\odot}$ \citep{tc99}. 
%However, several massive ($\sim2M_{\odot}$) pulsars have been found in recent
%years (J1614$-$2230 1.97$\pm0.04M_{\odot}$\citealt{dpr+10}; J1913+0327
%1.67$\pm0.02M_{\odot}$ \citealt{fbw+11}, J0348+0432 2.01$\pm0.04M_{\odot}$
%\citealt{afw+13}) that clearly out-lie this distribution.

%J1713+0747 appears to be nominal when compared with the early year samples
%despite of its recycling history.


\subsection{DM variation}
\label{sec:dmx}
The DM of a pulsar reflects the number of free electrons between
the pulsar and the telescopes and it varies because
our sight-line through the turbulent ISM and solar wind is changing as the
pulsar, the Sun, the Earth, and the ISM all move with respect to each other.
DM variation can affect the timing of high-precision pulsars significantly.

We fit simultaneously with other parameters the time-varying DM using the {\it DMX} model in {\sc tempo}.
This model fits independent DM values for ToA groups taken within 14 day
intervals, except for the L-band-only Mark~III ToAs. We grouped the Mark~III
ToAs together as a single group, because their frequency resolution and timing
precision are not sufficient for measuring epoch-to-epoch DM changes.

%The best-fit {\it DMX} model is consist of a DM time series over the length 
%of the data set.
%We made sure that in each group we have ToAs
%from at least two different bands (such as L-band and S-band). 
%The same procedure was carried out in
%\citet{dfg+13}.  This DM fitting was conducted with ToAs from all
%instruments except Mark~III, which produced only single-frequency ToAs.

%The grouping is feasible because of the frequency of our multi-band
%observations.
%The Arecibo 1400-MHz and 2300-MHz observations of PSR~J1713+0747 almost
%always happen consecutively on
%the same day, while the GBT 800-MHz and 1400-MHz observations often take place 
%on different days spaced by only few days apart.
%Therefore, the DM values in our {\it DMX} model were determined by ToAs from
%similar epochs and from at least two frequency bands. 
Figure \ref{fig:dmx} shows the measured DM variation of PSR~J1713+0747.
%Note that the systematic drop of DM around 2009 is likely instrumental, due to
%the use of different standard pulse profile templates when extracting 
%ToAs from the Mark~IV, ABPP data. The DM values inferred from different
%profile templates are expected to have a constant offset.
The sudden dip and recovery of DM around 2008 (MJD 54800) is 
due to changes either in the ISM or in the solar wind. This DM dip is also
observed independently by the Parkes observatory \citep{kcs+13}.
There is clear red-noise-like variance in the pulsar's DM with an RMS of
$\sim10^{-4}$ pc~cm$^{-3}$. 
%This DM variation may lead to ``red'' timing residuals as discussed in \citealt{kcs+13}.


%Note that the first part of this DM curve (MJD$<53200$) in
%Figure \ref{fig:dmx} has much larger uncertainties than the second half.
%This is because the relative inferior bandwidth of the Mark~IV and ABPP
%data. The Mark~IV data in particular have much smaller frequency bandwidth 
%compared to the later data (Table \ref{tab:obs}).
The ABPP ToAs are epoch-averaged ToAs, meaning they are collapsed in frequency to form one ToA per band per observation epoch, thus losing frequency resolution in the respective band. 
This averaging was done using a previous timing solution 
\citep{sns+05}. Thus the ABPP ToAs may be slighted biased to favor the
timing parameters in the previous solution. We also included the original
before-epoch-average Mark~IV ToAs, which were taken in the same epochs of the
ABPP data. Both of the Mark~IV and ABPP  ToAs are well fitted by our new
timing solution.


Spectrum analysis of the time variation of flux, pulse arrival phase, and DM have 
been employed to study the turbulent nature of the ISM since \citealt{cpl86, rl90}.
It has been shown that DM variations of some pulsars are consistent with
those expected from an ISM characterized by a Kolmogorov turbulence spectrum
\citep{cwd+90, ric90, ktr94, yhc+07, kcs+13, fst14}. In those cases one can calculate the 
structure function of the varying DM: 
\begin{equation}
D_{\phi}(\tau)=\left(\frac{2\pi K}{f^2}\right)\langle [DM(t+\tau)-DM(t)]^2\rangle, 
\end{equation}
where $\tau$ 
is a given time delay, $K=4.148\times10^3$~MHz$^2$pc$^{-1}$cm$^3$s, and $f$ is 
the observing frequency in MHz. We expect, under the simplest assumptions, 
this function to follow a Kolmogorov power law $D_{\phi}(\tau)=(\tau/\tau_0)^{\beta -2}$, 
where $\beta=11/3$ and $\tau_0$ is a characteristic time scale related to 
the inner scale of the turbulence. The pulsars with DM variations that fit this
theory generally have large DM variations on timescale of 
years. However, PSR J1713+0747 does not show significant long-term DM variation 
(Figure \ref{fig:dmx}). Conversely, it went through a steep drop and recovery 
around 2008. If such rapid DM changes are the result of variations in the ISM along
light of sight. Such ISM variations do not fit the general characteristics of Kolmogorov median. 
%Therefore the overall long-term variation is smaller
%compared with the 2008 event. 
%As a result, its structure function (Figure \ref{fig:dmx}) 
%follows a flatter spectrum than a Kolmogorov one.

DM variation could be studied using alternative methods such as Gaussian processes 
\citep{vv14a} or optimal filtering \citep{lbj+14}, but that is out of the scope of 
this paper.

\subsection{Pulsar spin irregularity}
\label{sec:spin}
%The pulsar timing residuals (Figure \ref{fig:res}) are the residuals left
%after we fit the ToAs with our timing model. 
%These residuals are likely caused by timing noise, DM variations, radiometer 
%noise, and other sources of noise. This section discusses PSR~J1713+0747's
%timing noise, i.e. ToA variations due to spin irregularities.

The term ``timing noise'' in pulsar timing generally refers to the non-white
noise left in the timing residuals.
An important part of these timing noise is expected to come from the pulsar's spin
irregularity, i.e. its long-term deviation from a simple linear slow down. 
Spin irregularity is often significant in younger pulsars, and often
requires modeling with significant high-order frequency polynomials (such as $\ddot{\nu}$, where $\nu$ is the pulsar's spin frequency). 
Potential causes of irregular spin behavior include unresolved
micro-glitches, internal superfluid turbulence, magnetosphere variations, or external torques caused by matter surrounding the pulsar \citep{hlk10, ymh+13, ml14}.
These mechanisms could lead to accumulative random perturbations in the 
pulsar's pulse phase, spin rate, or spin-down rate. 
\citet{sc10} pointed out that one could model these types of timing noise using random walks.
Random walks in phase (RW$_0$) would grow over time ($T$)
proportionally to $T^{1/2}$, and random walks in $\nu$ grows proportionally to $T^{3/2}$, random walk in
$\dot{\nu}$ grows proportional to $T^{5/2}$.
Such spin noise would likely have a steep power spectrum with more power in
the lower frequencies, also known as a ``red'' power spectrum. This
is considered as one of the main sources of ``red'' noise in pulsar timing.


The timing noise of radio pulsars has been studied by
\citet{ch80,cd85,antt94,dmhd95, mtem97}, and later by \citet{hlk10} and
\citet{sc10} with large samples. 
%\citet{antt94} characterized the significance of timing noise using the second
%spin derivative $\ddot{\nu}$. 
%This is because for most pulsars except the few youngest ones, 
%The expected $\ddot{\nu}$ from regular spin down is too small to be
%measurable, therefore the higher order spin parameters we observe are most
%likely the result of timing noise.
%They define the noise factor 
%\begin{equation}
%\label{eq:delta8}
%\Delta_8 = \log_{10}\left(\frac{1}{5\nu}|\ddot{\nu}|t^3\right).
%\end{equation}
%Here $t=10^8$~s$\sim 3$~yr is a fiducial time scale close to their average
%observation time span.
%Similarly, 
\citet{mtem97} adopted a generalized Allen Variance (traditionally used in
measuring clock stability) to characterize the timing instability of pulsars:
\begin{equation}
\label{eq:sigmaz}
\sigma_z(\tau) = \frac{\tau^2}{2\sqrt{5}}\langle c^2 \rangle^{1/2},
\end{equation}
where $\langle c^2\rangle$ denotes the sum of squares of the cubic
terms fitted to segments of length $\tau$. 
$\Delta_8$ is the logarithm of a dimensionless form of $\sigma_z(10^8~\rm s)$.
\citet{hlk10} found a best-fit scaling model of $\sigma_z({\rm 10~yr})$ 
from 366 pulsars:
\begin{equation}
\label{eq:hlk10}
\log_{10}[\sigma_z({\rm 10~yr})] =
-1.37\log_{10}[\nu^{0.29}|\dot{\nu}|^{0.55}]+0.52,
\end{equation} 
where $\nu$, $\dot{\nu}$ are the pulsar's spin and spin-down rate.
We find that \citet{hlk10}'s scaling model ($\sigma^{\rm model}_{z, \rm
10~yr}\simeq1\times10^{-12}$) over-predicted $\sigma^{\rm measured}_{z, \rm
10~yr}=5\times10^{-16}$ for PSR J1713+0747 by more than three orders of magnitude. 
%indicating that PSR~J1713+0747 has
%significantly less timing noise than average pulsars of similar spin behavior.

\citet{ch80} defined a different timing noise characteristic $\sigma^2_{\rm
TN,2}$ based on the root mean square of residuals $\sigma^2_{\msR,2}$ from a
timing fit that does not include any higher order spin parameters like
$\ddot{\nu}$. 
The timing noise term is related to $\sigma^2_{\msR,2}$:
\begin{equation}
\sigma^2_{\msR,2}(T) = \sigma^2_{\rm TN,2}(T) + \sigma^2_W, 
\end{equation}
where $\sigma^2_W$ is a time-independent term caused by white 
noise in the data.
In this definition, timing noise $\sigma^2_{\rm TN,2}(T)$ grows bigger over
time while white noise stays constant.  

\citet{sc10} studied the $\sigma^2_{\rm TN,2}$ from a large sample of pulsars
including canonical pulsars (CPs) and millisecond pulsars. They found a scaling
model:
\begin{equation}
\label{eq:sc10}
\ln(\hat{\sigma}_{\rm TN,2}) = 1.6 - 1.4\ln(\nu) +
1.1\ln|\dot{\nu}_{-15}|+2\ln(T_{\rm yr}),
\end{equation}
where $\dot{\nu}_{-15}$ is $\dot{\nu}$ in units of $10^{-15}$s$^{-2}$, and $T_{\rm yr}$
is the observation time span in years.
This scaling model predicts that, for 21-yr timing of PSR~J1713+0747, the
residual RMS without removing timing noise $\sigma^2_{\rm TN,2}$ would be
$\sim0.4$~$\mu $s. The measured RMS of the red noise residual 
$\sigma^2_{\msR,RN}=364$~ns, is consistent with the extrapolation
from \citet{sc10}.  
However, the best-fit scaling law also indicates that the residuals of the
sampled pulsars $\hat{\sigma}_{\rm TN,2}$ seem to grow linearly with $T_{\rm yr}^2$. 
If the timing noise of the sampled pulsars are due to the accumulation of 
spin noise, and the spin noise is caused by the same physical processes,
then this RMS growth rate would imply that the spin noise of pulsars has a
frequency power spectrum of power-law index $\gamma_{\rm red}\simeq-5$. This 
spectral index is consistent with the $\gamma_{\rm red}$
from our noise model (Figure \ref{fig:res}).

It is inconclusive whether or not the observed red noise can be interpreted as pulsar spin irregularity.
Other sources of noise also could have contributed significantly.
If we do assume that they are from spin irregularity, 
the estimated maximum likelihood red noise spectral index of $\sim-3$ 
favors that the pulsar spin irregularities come from
random walks in either spin phase or spin rate, although other explanations cannot be ruled out due to the
substantial uncertainty on the red noise spectral index (Figure \ref{fig:res}).

Finally, \citet{sc10} showed that the significance of timing noise coming from
gravitational wave (GW) background could be estimated as
$\sigma_{\rm GW,2} \approx$~1.3~ns~$A_0(T_{\rm yr})^{5/3}$, where $A_0$ is the
characteristic strain at $f=1$~yr$^{-1}$ and $T_{\rm yr}$ is the observational
time span in years. The current best upper limit on GW characteristic 
strain is $2.4\times10^{-15}$ \citep{src+13}, which predict an upper limit on
timing noise of $\sim500$~ns from GW background. Therefore, we
cannot rule out the contribution of gravitational waves in the timing noise.


\subsection{Pulse profile evolution in frequency}
\label{sec:FD}
After removing the dispersion that causes ToA delays proportional to $f^{-2}$,
where $f$ is the observing frequency,
 we still see small remaining frequency-dependent residuals from wide-band
observations using
different instruments and telescopes (Figure \ref{fig:FD}).  
It appears that the low-frequency ($\sim$800~MHz) signals lead the
high-frequency (L and S band) signals by tens of microseconds.
The cause of such profile evolution is not clear. It could be a change in the pulsar's
radiation pattern with frequency. Pulsar radiation of different frequencies may originate from
different parts of the star's magnetosphere, and 
the radiation region of the pulsars' magnetosphere may be slightly distorted,
leading to a frequency-dependent radiation pattern. \citet{pdr14} and \citet{ldc+14} 
extensively discussed this phenomenon and developed ToA extraction techniques
based on phase-frequency 2-D pulse profiles matching. This technique is not
yet applied to our dataset.

\citet{sns+05} allowed an arbitrary offset between ToAs taken with different
observing systems and at different frequencies.
However, the number of frequency
channels has increased by a factor of ten with the modern wide-band
instruments, making it a lot harder to mitigate profile-frequency evolution using jumps. 
Instead, we used the {\it FD} model, a polynomial of the logarithm of
frequency (Demorest et~al.\ in prep.; solid line in Figure
\ref{fig:FD}) to fit for and removed the profile-frequency
evolution. This model successfully removes the extra
frequency-dependent residuals and it requires only four parameters in the
case of PSR~J1713+0747.


\section{Results}
\label{sec:res}

\subsection{Intrinsic orbital decay}
\label{sec:obdecay}
We have measured an orbital decay from PSR~J1713+0747, $\dot{P}_{\rm b} =
0.36\pm0.17\times10^{-12}$s~s$^{-1}$ (Table \ref{tab:par1} and \ref{tab:par2}).
%Such a precise measurement can be used to test Gravitational wave radiation theories of Einstein's and beyond.
This orbital decay may not be intrinsic to the pulsar binary, but rather the
result of the kinetic motion between the binary and the
observer, i.e. a relativistic effect caused by differential 
acceleration in the Galactic gravitational potential
\citep{dt91} and a relativistic effect caused by the proper motion of the
pulsar. The relativistic effect
of proper motion is also known as the ``Shklovskii'' effect (
\citealt{shk70}). Luckily, we have good measurements of the distance and proper
motion of the binary system, which allow us to remove these effects and study the system's intrinsic orbital decay.
\begin{equation}
\dot{P}_{\rm b}^{\rm Gal} = \frac{A_{\rm G}}{c} P_{\rm b} =
-0.10\pm0.02\times10^{-12}~{\rm s~s^{-1}}
\end{equation}
where $A_{\rm G}$ is the line-of-sight acceleration of the pulsar binary;
this term is dominated by the difference in the Galactic accelerations of the
binary and our solar system, and is obtained using
Equation 5 in \citet{nt95}, Equation 17 in \citet{lwj+09} and the Galactic
potential model by \citet{hf04a}.
On the other hand, the Shklovskii effect causes $P_{\rm b}$ to
change by
\begin{equation}
\dot{P}_{\rm b}^{\rm Shk} = (\mu_{\alpha}^2+\mu_{\delta}^2)\frac{d}{c}P_{\rm
b} = 0.65\pm0.02\times10^{-12}~{\rm s~s^{-1}}.
\end{equation}
Therefore, the pulsar's intrinsic orbital decay is $\dot{P}_{\rm b}^{\rm Int}
= \dot{P}_{\rm b}^{\rm Obs} - \dot{P}_{\rm b}^{\rm Shk} - \dot{P}_{\rm b}^{\rm
Gal} = (-0.20\pm0.17)\times10^{-12}$s~s$^{-1}$, and is consistent with zero.

Due to the very long $\sim$68 day orbit, the binary's decay due to the
emission of gravitational
radiation is expected to be undetectable: $\dot{P}_{\rm b}^{\rm GR} =
-6\times10^{-18}$s~s$^{-1}$ \citep{lk05}.  Therefore, the insignificant
intrinsic orbital decay rate is entirely consistent with the
description of quadrupolar gravitational radiation within General
Relativity (GR).

Other than the gravitational radiation, two other effects could have played a role in
$\dot{P}_{\rm b}^{\rm Int}$. One, $\dot{P}_{\rm b}^{\dot{M}}$, is caused by mass loss in the
binary system, and the other, $\dot{P}_{\rm b}^{\rm T}$, is the contribution
from tidal effects.
The pulsar and the white dwarf both could lose mass due to their magnetic dipole radiation; the maximum
mass loss rate due to this effect can be estimated from the
star's rotational energy loss rate. In the case of the pulsar, $\dot{M}_{\rm
PSR}=\dot{E}/c^2$, measurable through the spin down rate of the pulsar.
The white dwarf generally loses mass at a much lower rate than the pulsar.
%\begin{equation}
%\dot{P}_{\rm b}^{\dot{m}} = 8\pi^2\frac{I_{\rm
%PSR}}{Mc^2}\frac{\dot{P}}{P^3}P_{\rm b} \sim 10^{-16},
%\end{equation}
%where $M=M_{\rm PSR} +M_{\rm WD}$ is the total mass of the system and
%$I\sim10^{45}$g~cm$^2$ is the angular momentum of inertia.
Therefore, orbital change due to mass loss can be estimated as $\dot{P}_{\rm
b}^{\dot{M}}\sim 1\times10^{-14}$s~s$^{-1}$ (\citealt{dt91}; Equation 9 and 10
of \citealt{fwe+12}). This is an order of magnitude smaller than the measured
uncertainties on $\dot{P}_{\rm b}^{\rm Int}$.
The tidal effect in this binary system is expected to be $\dot{P}_{\rm b}^{\rm
T}\ll1\times10^{-14}$s~s$^{-1}$ based on the most extreme scenarios (the white
dwarf spins at its break-up velocity and the tidal synchronizing time scale equals the
characteristic age of the pulsar; see Equation 11 in \citealt{fwe+12} and
references therein).
Both of these extra terms are much smaller than the observed uncertainties
on $\dot{P}_{\rm b}^{\rm Int}$.
%The tidal effect can be estimated to $\sim$ according to Equation 11 of
%\citealt{fwe+12} and references therein. 


\subsection{Time Variation of $G$}
\label{sec:Gdot}


%The wide orbit of PSR J1713+0747 ($\sim$68 days) makes it 
%some alternative theories of gravity, such as scalar-tensor gravity,
%predict larger orbital decays.
%Consequently, measurements of pulsar binaries' ``excess'' orbital decay
%$\dot{P}_{\rm b}^{\rm exc}=\dot{P}_{\rm b}^{\rm Int} - \dot{P}_{\rm
%b}^{\dot{M}}  - \dot{P}_{\rm b}^{\rm T} - \dot{P}_{\rm b}^{\rm GR}$ have been
%used to constrain alternative theories \citep[e.g.][]{lwj+09, fwe+12}. 
Based on the measurement of the ``excess'' orbital decay 
$\dot{P}_{\rm b}^{\rm exc}=\dot{P}_{\rm b}^{\rm Int} - \dot{P}_{\rm
b}^{\dot{M}}  - \dot{P}_{\rm b}^{\rm T} - \dot{P}_{\rm b}^{\rm GR}$,
\citet{dgt88} derived a generic phenomenological limit for $\dot{G}$: 
$\dot{G}/G\simeq-\dot{P}_{\rm b}^{\rm exc}/(2P_{\rm
b})=1.0\pm2.3\times10^{-11}$~yr$^{-1}$ using the timing of binary PSR~B1913+16. 
Since then $\dot{P}_{\rm b}^{\rm exc}$ of pulsar binaries, including 
PSR~J1713+0747, have been used to 
constrain $\dot{G}/G$ \citep{ktr94, nss+05, dvtb08, lwj+09, fwe+12}. 
So far all pulsar observations show $\dot{G}/G$ consistent with being zero, with 
upper limits largely determined by the uncertainties in orbit decay rate, distance, 
and proper motions.
PSR~J1713+0747 has the smallest known $\dot{P}_{\rm b}^{\rm exc}/(2P_{\rm
b})\simeq(-0.5\pm0.9)\times10^{-12}$~yr$^{-1}$ (Section \ref{sec:obdecay}) and is
particularly useful for constraining the time variability of gravitational
constant.

\citealt{lwj+09} and \citealt{fwe+12} showed that $\dot{G}/G$ can be constrained by pulsar
binaries in a more rigorous fashion
by incorporating a class of realistic alternative gravitational theory --- the 
scalar-tensor theory.
In the framework of this theory,
the scalar field that interacts with the mass changes over time as the
Universe expands. This change will cause the local value of
$G$ to vary. The changing $G$ will also change the orbital period of a binary system:
\begin{equation}
\dot{P}_{\rm b}^{\dot{G}} = -2 \frac{\dot{G}}{G}
\left[1-\left( 1+\frac{m_c}{2M}\right) s_p\right]P_{\rm b},
\end{equation}
where $s_p$ is a dimensionless factor characterizing the self-gravity
``sensitivity'' of the pulsar \citep{nor90}.
This formalism is only slightly different from the generic form of
\citet{dgt88}.

Meanwhile, in the framework of an alternative gravitation theory that violates
SEP, a binary system may emit dipole gravitational radiation (\citealt{Will93, Will01, lwj+09, fwe+12} and references
therein). Such effects arise when the two bodies are very different in terms
of their self-gravity, i.e.  their compactness.
This extra dipole radiation could lead to an extra orbital decay term:
\begin{equation}
\dot{P}_{\rm b}^{\rm D} \simeq -4\pi\frac{T_{\odot}\mu}{P_{\rm b}}\kappa_D S^2,
\end{equation}
\citep{lwj+09}, where $T_{\odot}=G{\rm M_{\odot}}/c^3=4.925490947$~${\rm
\mu}$s \citep{lk05}, $\mu$ is the reduced mass ($m_pm_c/M$) of the system, $\kappa_D$ is dipole
gravitational radiation ``coupling constant'', and $S$ is the difference
between the self-gravity ``sensitivity'' of the two bodies ($S = s_p - s_c$;
$s_p\sim0.1m_p/M_{\odot}$ according to \citealt{de92} ; and $s_c\ll s_p$).
In Einstein's general relativity $\kappa_D=0$ --- there is no self-gravity induced
dipole gravitational radiation, but it is often not the case in alternative
theories that violate the SEP.

PSR~J1713+0747 has a wider binary orbit than most other
high-timing-precision pulsar binaries, making its $\dot{P}_{\rm b}^{\rm D}$
very small. Conversely, $\dot{P}_{\rm b}^{\dot{G}}$ is larger when $P_{\rm b}$
is large. This makes PSR~J1713+0747 the best pulsar binary system for constraining
the effect of the changing gravitational constant $\dot{G}$. Limits 
on both $\dot{G}$ and $\kappa_D$ can be estimated in the same fashion as in
\citet{lwj+09}: by solving $\dot{G}$ and $\kappa_D$ simultaneously 
from the equation $\dot{P}_{\rm b}^{\rm exc} = \dot{P}_{\rm b}^{\rm D} +
\dot{P}_{\rm b}^{\dot{G}}$ (Equation 29 of \citealt{lwj+09}) of different
pulsars. We applied this method to four pulsars: PSR J0437$-$4715, PSR J1012+5307, PSR
J1738+0333, and PSR~J1713+0747 using timing parameters reported in
\citet{lwj+09}, \citet{fwe+12}, and this work.
The resulting confidence region of $\dot{G}$ and $\kappa_D$ is shown in Figure
\ref{fig:Gdot}.
We found, at 95\% confidence limit, $\dot{G}/G =
0.6\pm1.1\times10^{-12}$~yr$^{-1}$; $\kappa_D=-0.9\pm3.3\times10^{-4}$. 
This constraint on $\dot{G}$ is more stringent than
previous pulsar-based constraints \citep{fwe+12},
and close to one of the best constraints of this type
($\dot{G}/G=-0.07\pm0.76\times10^{-12}$~yr$^{-1}$) from the Lunar Laser Ranging
(LLR)
experiment \citep{hmb10}, which measured Earth-Moon distance to $\sim10^{-11}$
precision through 39 yrs of laser ranging.
\citet{fle+14} showed that $\dot{G}/G$ can be constrained to 
$0.009\pm0.182\times10^{-12}$~yr$^{-1}$ through the analysis of solar system planetary ephemerides.
The pulsar-timing $\dot{G}$ and $\kappa_D$ limits are particularly interesting 
in the framework of the SEP-violating alternative theories, because they are from 
a test using objects of strong self-gravitation.

\subsection{Strong equivalence principle and Lorentz invariance}
\label{sec:sep}
General relativity is the only gravitation theory that satisfies
the SEP, which states that the gravitational
effect on a small test body is independent of its constitution. Specifically,
this principle holds that bodies of different self-gravitation should behave the same in
the same gravitational experiments. This principle is violated in alternative
theories of gravitation like the aforementioned Jordan-Brans-Dicke
scalar-tensor theory. The PSR~J1713+0747 binary is an excellent laboratory for testing 
effects of SEP violation. If the SEP is violated, the neutron star and the white
dwarf will be accelerated differently by the Galactic gravitational field, causing
the binary orbit to be polarized toward the center of the Galaxy. The excess 
eccentricity is expected to be (\citealt{ds91}):
\begin{equation}
|\textbf{\textit{e}}_F| = \frac{1}{2}\frac{\Delta\, g_{\bot}
  c^2}{G(M_{\rm PSR}+M_{\rm
c})(2\pi/P_{\rm b})^2},
\end{equation}
where $g_{\bot}$ is the projection of Galactic acceleration on the orbital plane 
and $\Delta$ is the dimensionless factor that characterizes the significance 
of SEP violation. A recent model of the Galactic acceleration can be found in
\citet{hf04a}.

Lorentz invariance is another principle that is satisfied by GR and may
be violated by alternative theories. This principle
states that there is no preferred inertial reference frame. The violation of
Lorentz invariance would lead to polarization of binary orbits along the
direction of a preferred frame.
One can estimate such an effect in a strong-field version of the Post-Newtonian Parameterization
(PPN) framework \cite{de92}; the excess eccentricity is expected to be (\citealt{bd96}):
\begin{equation}
|\textbf{\textit{e}}_F| = \hat{\alpha}_3 \frac{c_p|\textbf{\textit{w}}|P_{\rm b}^2}{24\pi P}
\frac{c^2}{G (M_{PSR}+M_{\rm c})}\sin \beta,
\end{equation}
where $\textbf{\textit{w}}$ is the absolute velocity of the binary system
relative to the preferred frame of reference, typically taken as that of the cosmic microwave background (CMB), $P$ is the pulsar's spin period, $\beta$ is the
angle between $\textbf{\textit{w}}$ and the spin axis of the pulsar, and
$\hat{\alpha}_3$ is the strong-field version of one of the Post-Newtonian
parameters that characterizes the violation of Lorentz invariance. 
Here $\textbf{\textit{w}} = \textbf{\textit{w}}_{\odot} + \textbf{\textit{v}}_{\rm PSR}$, where
$\textbf{\textit{w}}_{\odot}=384\pm139$km~s$^{-1}$ is the velocity of
the solar system relative to the CMB (\citealt{aaa+13}),
and the term $\textbf{\textit{v}}_{\rm PSR}$ is the relative speed of the pulsar to our solar system. $\textbf{\textit{v}}_{\rm PSR}$ is only partially known because we can measure the pulsar's
proper motion on the sky but we cannot measure its radial velocity.

Fortunately, many variables in these equations are measurable in the
case of the PSR~J1713+0747 binary. This makes it possible to constrain $\Delta$
and $\hat{\alpha}_3$ using Bayesian techniques 
%assuming certain fiducial priors for the unmeasurable variables like the radial component of
%$v_{\rm PSR}$ 
\citep{sns+05, sfl+05, gsf+11}. Based on our 21-yr 
timing of J1713+0747 alone, we find 95\% confidence limits on the violations of SEP and
Lorentz invariance $\Delta < 0.01$ and $\hat{\alpha}_3<2\times10^{-20}$, 
slightly improving the single pulsar limits from earlier data on this pulsar
\citep{sns+05, sfl+05, gsf+11}.
Stronger limits can be found by combining the results from
multiple similar pulsar systems \citep{wex00,sfl+05, gsf+11}.

%\section{Testing fundamental physics principles}
%\subsection{Conservation of momentum}
%\subsection{The Strong Equivalence Principle}

\section{Summary}
In this paper, we present a comprehensive model of high precision timing observations of
PSR~J1713+0747 that spans 21 yrs. 
We improved measurements on the pulsar and its companion's masses and the
shape and orientation of the binary orbit. We also detect, for the first time, an apparent
decay of the orbit due to Galactic differential accelerations and the Shklovskii effect.
These measurements, when combined with those of other pulsars, 
significantly improve the pulsar timing limit on the rate of change of the gravitational
constant, $\dot{G}$. Although the pulsar constraint is not better than the
best solar system ones, it is nevertheless an independent test using 
extra-solar binary systems thousands of light-years away. The pulsar tests
also could be more constraining for some
variances of scalar-tensor theories that predict stronger non-GR effects in
strong-field regime.
The new best pulsar timing limit on $\dot{G}/G$ is 
$0.6\pm1.1\times10^{-12}$~yr$^{-1}$ ($<0.033H_0$ based on the 3-$\sigma$ limit), where $H_0$ is the Hubble constant. 
In other words, the change rate of gravitational constant has to be a factor
of at least $31$ (3-$\sigma$ limit) slower than the average expansion rate of
the Universe.
%Although this $\dot{G}/G$ limit is derived in the
%framework of scalar-tensor theories, it is a more rigorous and conservative
%limit than the generic phenomenological limit 
%$\dot{G}/G\simeq-\dot{P}_{\rm b}^{\rm exc}/(2P_{\rm
%b})=4\pm7\times10^{-13}$~yr$^{-1}$ \citep{dgt88}.

%Because $\dot P_{\rm b}$ is a second order term in the timing 
%model, its measured precision improves very fast, $\sigma \dot
%P_{\rm b} \propto T^{5/2}$, where $T$ is the total observational time
%span \cite{dt92}.
%It is possible that the precision of the above gravitational experiment from pulsar
%timing may surpass the precision of the LLR experiment in the future and
%become the best constraint of this type.

Meanwhile, the precise measurements of PSR J1713+0747's orbital eccentricity and
3-D orientation allow us to test the violation of SEP and 
Lorentz invariance with it. We found a single-pulsar 95\% upper limit on 
$\Delta <0.01$, the SEP violation factor, and
$\hat{\alpha}_3<2\times10^{-20}$, the PPN parameter that characterizes
violation of Lorentz invariance. 
Because of the different statistical analysis methods were used, our 
$\Delta$ and $\hat{\alpha}_3$ limits  are slightly
different but still consistent with the results of the same tests in previous publications 
\citep{wex00, sns+05, sfl+05, gsf+11}.
Ultimately, the best test on SEP violation could come from the newly
discovered pulsar triple system PSR J0337+1715 \citep{rsa+14}. In this case 
the inner pulsar-white dwarf binary is orbited by another white dwarf in an
outer orbit, making this system a great laboratory for testing
the free fall of a neutron star and white dwarf in external gravity field.

We studied the time variation of PSR~J1713+0747's DM from 1998 to 2013, and
fitted the structure function of the DM variation with a power law.  
The best-fit power law index is 0.49(5), significantly smaller than the 5/3 
index expected from a ``pure'' Kolmogorov medium. This relatively ``flat'' structure
function could be the result of either the lack of long-term DM variations or an
excess of short-term variations. The sudden DM dip around 2008 (Figure
\ref{fig:dmx}) is a good example of such short-term DM variations.
Similar non-Kolmogorov DM variations were observed from some of the
other NANOGrav pulsars (Levin et al. \ in prep.). Evidence of non-Kolmogorov behavior 
in the ISM was also found in the analysis of multi-frequency pulsar scatter times \citep{lkk15}.

In conjunction with the timing modeling, we also modeled observational noise
such as jitter and red noise
 using the GLS fit and a covariance matrix that included the
correlated and uncorrelated noise terms.
We found that our timing result is significantly affected by the noise
model, especially the jitter noise, suggesting that the adoption of jitter
modeling may be necessary in cases of high precision pulsar timing. 
We found that our noise parameters and timing residuals are consistent with the jitter
noise estimates from \citet{sc12} and the timing noise estimate from \citet{sc10}. However,
the scale law extrapolated from large sample studies of timing noise in \citet{hlk10}
overestimated the timing noise level $\sigma_z({\rm 10 yr})$ in this pulsar.
%We also detected a red noise signal of spectral index $\sim-3$ that could be
%the pulsar's spin noise. The measured spectral index favors that the spin
%noise is composed of random perturbations to either the pulsar's spin rate
%$\nu$ or spin-down rate $\dot{\nu}$, although other explanations could not be
%ruled out.

Our noise model parameters and timing residual RMS (Table \ref{tab:wrms})
provide a crude estimation of the amount of noise in our data. The weighted
root mean square (WRMS; see Table \ref{tab:wrms} for definition) of
the 21-yr daily-averaged timing residuals is $\sim 92$~ns. 
Table \ref{tab:wrms} shows a systematic improvement in the timing accuracy of
this pulsar in the last two decades, due to the advances in instrumentation.
But the improvements are not as large as expected from the radiometer 
equation, perhaps because of pulse jitter. 

The reconstructed red noise signal is most prominent in the Mark~IV and ABPP data
but not significant in the ASP, GASP, GUPPI, and PUPPI data.
This is consistent with the non-detection of red noise in the five years
ASP and GASP data by \citet{pjl+13} using an auto-correlation technique.
It is not clear if the red noise processes in Mark~IV and ABPP data
are from pulsar spin noise, because we cannot rule out other sources such as
instrumental drift.
Assuming that the red noise is caused by spin irregularity,
the best-fit spectrum index is consistent with
spin irregularity that was caused by random walks in
either the spin phase or the spin rate of the pulsar, but it does not exclude other explanations
due to its large uncertainty (see bottom panel of Figure \ref{fig:res}).
%A small part of the red noise [$<\sim 10\%$; based on the gravitational wave background
%upper limit of xxx \citep{src+13}] could have been caused by gravitational wave signals.
The observed red noise level is also consistent with the prediction
from the current best upper limit of GW background \citep{src+13}, therefore,
we cannot rule out significant timing noise contribution from GW background. 


