
\begin{abstract}
Pulsars are precise cosmic clocks and excellent testing grounds for
fundamental physics, especially for testing gravitational theories. 
We report 21-year timing of one of the most precise
pulsars: PSR~J1713+0747. The pulse times of arrival of this pulsar are well
modeled by a comprehensive binary model, which included the 3-D orientation 
of the orbit in the sky, with residuals of estimated variance of $\sim 118$~ns. 
The new dataset allows us to update and greatly improve previous measurements
of the system properties, including the masses of the neutron star
[1.3(1)~\Msun] and white dwarf [0.290(14)~\Msun] and the parallax
 [$\pi=$0.83(3)~mas].  
%We were able to measured for the first time a statistically significant value for the 
%binary orbit's apparent decay due to motion between the pulsar and the observer. 
We have measured a change in the observed orbital period of PSR~J1713+0747. We
attribute this to the relative motion of the pulsar system and the Earth, and
we find that the intrinsic change in orbital period is $\dot{P}^{\rm Int}_{\rm
b}=-0.13\pm0.18\times10^{-12}$s~s$^{-1}$, not distinguishable from zero.
We use this to place limits on changes in the gravitational constant $\dot{G}$. 
$\dot{G}/G$ is found to be $-0.3(1.1)\times10^{-12}$~yr$^{-1}$ and at 
least a factor of $34$ (99.7\% confidence level) slower than the average expansion 
rate of the Universe. 
%Such a limit has important implications for alternative theories of gravitation.
\end{abstract}


\section{Introduction}
\label{sec:intro}
\linenumbers
We present 21-year timing of the millisecond pulsar PSR~J1713+0747. This
pulsar was discovered in 1993 \citep{fwc93}. It is one of the brightest pulsars timed by the
North American nano-Hertz Observatory for Gravitation Waves (NANOGrav;
\citealt{mcl13, dfg+13}), has the smallest timing residual of all NANOGrav
pulsars \citep{dfg+13}.
The timing analysis reported in this paper is pivotal to future pulsar timing array
projects, such as the estimation of single-pulsar upper limit of the stochastic gravitational wave background.
Observations of this pulsar were reported previously in
\citealt{cfw94}, \citealt{vb03}, \citealt{sns+05}, \citealt{hbo06} and \citealt{ver09}.
%The high timing precision and the long base line allowed us to precisely 
%measure the pulsar binary's orbit, masses, distance, proper motion, 
%and orientation on the sky (see Section \ref{sec:model} for
%details). 

Millisecond pulsars (MSPs) are very stable rotators due to their enormous
angular momentum. PSR~J1713+0747 is a MSPs residing in a wide binary orbit with a white dwarf companion (Section \ref{sec:model}). 
The latest wide-bandwidth and high-resolution instruments allow us to model and account for the variation in the interstellar medium (ISM)
(Section \ref{sec:dmx}) and small distortions of the pulsar's magnetosphere (Section \ref{sec:FD}). 
The pulse arrival times of the pulsar are well fit by a binary model with
a nearly circular orbit that is edge-on to our line of sight. The masses of
the binary can be inferred through the measurement of Shapiro delay \citep{sns+05}. 
The system's distance is well-measured through a timing parallax. We detect
a changing projected orbital semi-major axis due to the orbit's proper motion
on the sky. Through the change rate of projected semi-major axis, we can infer 
the orientation of the orbit in the sky.
This is one of the only few binaries in which the 3D-orientation of the
binary orbit can be completely solved. Using 21 years of data, we
refine the previously-published measurements of these orbital parameters.
We also find strident constraint on the decay rate of the binary orbit due to the system's
motion (the Shklovskii effect and Galactic differential acceleration).
%These measurements also enable better tests of the theories of gravitation (see Section \ref{sec:res} for details).

The stability and long orbital period of the PSR~J1713+0747 binary make it the
perfect laboratory for observing the smallest cosmic variance, 
one such as the time variation of Newton's gravitational ``constant'' $G$. 
This interesting conjecture of $G$ varying on a cosmological timescale was first 
raised by \citet{dir37} based on his large-number hypothesis, and 
revisited later by gravitational theories involving extra dimensions
\citep{mar84,ww86a}.
In this paper, we seek to constrain the time variation of $G$ using this pulsar
binary in the framework of one of the last remaining alternative theories of gravitation, the scalar-tensor theory \citep{jor59,fie56,bd61}. 
This theory modifies Einstein's equation of gravitation by coupling mass with
a large-scale scalar-tensor field, and predicts that, as the universe expands,
the scalar field will also expand, causing the gravitational constant to vary  
in the cosmological timescale. 
PSR~J1713+0747 is likely the best pulsar binary for testing the constancy of
$G$ thanks to its high timing precision and long orbital period. Using timing
results reported in this paper, we found a stringent upper limit on the
$\dot{G}$ (see Section \ref{sec:Gdot} for detail). 
%like Luna Earth binary or pulsar
%binaries in the matter of decades. The Luna Laser Ranging (LLR) experiment measured the
%Luna-Earth distance to $10^{-11}$ precision for a 39 years time span, and thus put a very good upper limit on the change rate of gravitational constant
%\citep{hmb10}. Recently, thanks to the high-precision pulsar time array
%projects, measurements of pulsar binary orbits have also been improved
%greatly, reaching similar precision as the LLR experiment and start to provide
%meaningful independent constrains on $\dot{G}/G$. PSR J1713+0747 is uniquely
%useful in this particular test, because of its long orbital period. Unlike in
%other pulsar binaries, where the orbital decay could be affected by
%gravitational wave radiation (as predicted by Einstein's GR or by 
%alternative theories), this system has very small gravitational radiations. 
%Therefore, it is likely the best system for testing the constancy of
%gravitational constant. We found that by combining the timing result of
%J1713+0747 with that of other pulsars', we get a significantly 
%improved upper limit on $\dot{G}/G$ compare with previous timing-based 
%limits (see Section \ref{sec:Gdot} for detail).

The PSR~J1713+0747 binary is also an excellent laboratory for testing physical 
principles such as the strong equivalence principle (SEP) and the Lorentz invariance, because small violation of these principles could polarize the
binary orbit and result in potentially observable effects. We can set
upper limits on the violations of these principles by observing
low-eccentricity pulsar binary systems. Section \ref{sec:sep} presents the
constraint on the violation of SEP and Lorentz invariance from our 21 years of observation of PSR~J1713+0747.

\section{Observations}
Timing observations of PSR~J1713+0747 started in 1993 at the Arecibo
Observatory. \citet{sns+05} reported the results of the first 12 years
of timing observations. Since 2005 the pulsar is observed approximately monthly
at 1400~MHz and 2300~MHz by the Arecibo
observatory and at 800~MHz and 1400~MHz by the Robert C. Byrd Green Bank
Telescope (GBT).

The timing observations were conducted using multiple generations of pulsar
data backends (Table \ref{tab:obs}).
{\bfref
The earliest observations (1992-1994) used the Princeton Mark~III
\citep{skn+92}, which collects two polarization data with a filter bank of 32
spectral channels each 1.25~MHz wide. 
Observations between 1998 and 2004 used the Princeton Mark~IV
\citep{sst+00} instruments and the Arecibo-Berkeley Pulsar Processor
(ABPP) system \citep{bdz+97} in parallel.. 
The Mark~IV system collects 10~MHz 
passband data using 2-bit sampling. The
data were coherently dedispersed and folded at the pulse period offline.
The ABPP system samples voltages with 2-bit resolution and filter the passband 
into 32 spectral channels (1.75~MHz
per channel and 56~MHz in total for 1410~MHz band; 3.5~MHz per channel and 
112~MHz in total for 2380~MHz band), and apply coherent dedispersion to each
channel using 3-bit coefficients. 
In this paper, we used daily-averaged Mark~III, Mark~IV, and ABPP TOAs reported in \citet{sns+05}, because the original 190~s integration TOAs are not accessible.

From 2004 to 2011/12, pulsar data were collected with the Astronomical Signal
Processor (ASP; \citealt{dem07}) and its Green Bank counterpart GASP \citep{dem07}.
The (G)ASP systems records 64~MHz bandwidth data and apply real-time
coherent dedispersion and pulse period folding. The resulting data contain
2048-bin full-Stokes pulse profiles integrated over 1-3 minutes in 16 4~MHz channels. 
The J1713+0747 ASP/GASP data were also reported in \citet{dfg+13}.

We started using 
the Green Bank Ultimate Pulsar Processing Instrument (GUPPI; \citealt{GUPPI}) for GBT 
observations in 2010 and its clone the ``Puerto-Rican Ultimate Pulsar Processing Instrument''
(PUPPI) for Arecibo observations in 2012. 
GUPPI and PUPPI uses 8-bit sampling in real-time coherent dedispersion and
pulse period folding mode and produce 2048-bin full-Stokes
pulse profiles integrated over 10 second intervals.
When observing with 820~MHz receiver at GBT, We use GUPPI to collect data from 62 spectral 
channels each 3.125~MHz wide, covering 724-918~MHz in frequency. With L-band receiver, GUPPI
uses 58 spectral channels each 12.5~MHz wide, covering the 1150-1880~MHz band. 
When observing with L band receiver at Arecibo, we use PUPPI to collect data from 1150-1765~MHz
using 50 spectral channels each 12.5~MHz wide. When observing with the S band
PUPPI takes data from 1770-1880 MHz and 2050-2405 MHz using 38 spectral
channels each 12.5~MHz wide.
The spectral bandwidth and resolution provided by GUPPI and PUPPI is crucial for resolving the pulse profile evolution in frequency described in Section \ref{sec:FD}.

The date span, number of observation epochs, specifications of the
systems are listed in Table \ref{tab:obs}. }

We combine the pulse time of arrivals (TOAs) used in \citealt{sns+05} and those
of the later observations, for a data span of 21 years with a
noticeable gap between 1994 and 1998, during the Arecibo upgrade.
Data timestamps are derived from observatory masers and retrocorrected
to the Universal Coordinated Time (UTC) via GPS and then further
corrected to the TT(BIPM) timescale using the 2012 version BIPM clock corrections with extrapolations to 2013.
The TOAs are measured from the observational data through a series of
steps. First the data are folded, as they were being taken, into pulse
profiles using an ephemeris known to be good enough for predicting the
pulse periodicity for the duration of the observation. The folded
profiles from different frequency channels and sub-integrations are
often summed together to improve the S/N of the profiles.  Orthogonal
polarizations are summed to produce a total-intensity profile.
The summed profiles are then compared with a well-measured standard
pulse profile from the appropriate frequency band. We employ
frequency-domain cross-correlation techniques \citep{tay92} to determine the phase of the pulse peak relative to the midpoint of the observation. The final TOA of a summed profile is then calculated by adding the mid-observation time and the product of pulse period and the measured peak phase.
The flux density of the pulsar in these observations can also be
measured by comparing the signal strength in the data with that of a
calibration observation taken right before or after the pulsar
observation in which a signal with known strength was injected. For
the post-upgrade Arecibo data and all the GBT data, the
flux density of the calibration signal is calibrated every month by
comparing it with an astronomical object of known and constant flux
density, in this case, the AGNs J1413+1509 and B1442+09.


{\bfref 
The backends Mark~IV and ABPP are used in parallel for J1713+0747 
observations between 1998 and 2004. They collect data with different bandwidth (Table
\ref{tab:obs}). The Mark~IV TOAs were computed using
the same profile template used for later instruments like (G)ASP and (G)PUPPI. 
But the ABPP TOAs were computed using a different template.
To align the ABPP TOAs with Mark~IV, we
fitted a phase offset between the 1410~MHz TOAs of the two instruments, and found
that ABPP 1410~MHz TOAs trails that of Mark~IV by $0.46791205$ in pulse phase.
However, due to the difference in bandwidth and center frequencies, the S-band
ABPP TOAs trails Mark~IV slightly more than aforementioned, and the exact
offset value depend on our model of the profile evolution in frequency (FD
model; see Section \ref{sec:FD} for details). Therefore, in
addition to the aforementioned phase offset, we also 
fit the S band ABPP TOAs with an extra time offset as an unknown parameter.
%However, the same phase offset does not align the S band TOAs of Mark~IV and
%ABPP. 

During the transition from Mark~IV/ABPP to the ASP backend, Mark~IV
was kept running in parallel with ASP for several epochs. 
The overlapping Mark~IV data were not published in previous papers. 
In this work, we used these overlapping TOAs (labeled as Mark~IV-O in Table \ref{tab:obs}) 
and determined that the 1410~MHz TOAs of Mark~IV trails that of ASP by
2.28~$\mu$s. Consequently, the
1410~MHz TOAs from Mark~IV, ABPP, (G)ASP, and (G)PUPPI form a connected time
series spanning 16~years within which no unknown time offset has to be fitted. 
Unfortunately, there is no
over-lapping observations connecting these 16~yr data with the earliest  Mark~III TOAs. 
Time offsets have to be
fitted for the 1410~MHz Mark~III TOAs as unknown parameters in the timing
model. The resulting timing solution is still
clearly phase connected because the uncertainties on these time offset are
much
smaller than one spin period of the pulsar. However, these time offsets will
slightly inhibit our ability in detecting small variabilities in time scale longer 
than 16 years.

The backend instruments introduce varying degrees of computational
and electronic delays into the measured TOAs. 
As a result, the TOAs from different backends are different by small time
offsets.
In order to measure these offsets, we allow data to be taken 
with both backends in parallel during a period of time, usually about few years.
As a result, the offset between GASP and GUPPI, and the offset between ASP and
PUPPI are well measured from the these parallel observations using all NANOGrav
pulsars. These offsets are coded into the TOAs such that they do not affect
TOA modeling. 


}

\section{Timing model}
\label{sec:model}
%We model the pulsar's timing behavior with the comprehensive model described by \citet{sns+05},
%which included the effects of pulsar spin, astrometry, orbital motion,
%Shapiro delay and dispersion due to the ISM.
We employ the pulsar timing package {\sc tempo}
\footnote{\url{http://tempo.sourceforge.net}} to model the TOAs. 
The rotation of the pulsar is modeled with polynomials of spin frequency 
$\nu$ and $\dot{\nu}$ (Table \ref{tab:par}) in order to account
for the pulsar's spin, spin down.
The pulsar's position ($\alpha$, $\delta$) and proper motion ($\nu_\alpha$, $
\nu_\delta$) on the sky and its parallax $\pi$ is also measured through timing modeling. 
The distance of the pulsar can be inferred accurately from our parallax
measurement $D_{\rm PSR}=$1.12(2)~kpc. This distance is consistent with and
more accurate than the VLBA parallax distance of 1.05(6)~kpc \citep{cbv+09}.
We employed the \citet{dd86} (DD) model of binary motion to fit for the binary parameters, 
including the masses of
the neutron star ($M_{\rm PSR}$) and the white dwarf ($M_{\rm c}$) measured
through Shapiro delay (Section \ref{sec:mass}),
the orbital period $P_{\rm b}$, angle of periastron $\omega$, time of
periastron passage $T_0$, projected semi-major axis $x$ and its change rate
$\dot{x}$ due to proper motion of the orbit. 
Compared with previous timing efforts, we detected, for the first time, a
significant decay rate of binary period $\dot{P}_{\rm b}$ due to relativistic
effects. This is described in Section \ref{sec:obdecay}.    
The T2 model of binary motion in {\sc tempo2} \citep{hem06} also can be used for modeling this pulsar. We tested the T2 model and found that its best-fit binary parameters are physically consistent with those we found using DD model.
The DMX model was used to fit dispersion measure (DM) variations caused by changes in the ISM (see Section \ref{sec:dmx} for details). The FD model was used to model profile
evolution in frequency (see Section \ref{sec:FD} for details). 

{\bfref
In order to account for unknown systematics in TOAs from different
instruments, and epoch-correlated noise like pulse jitter noise from pulsar
magnetospheric activities. 
We employed a general noise model that parameterize both uncorrelated noise and
correlated noise. The noise modeling is discussed in Section \ref{sec:noise} and the
noise model parameters are listed in Table \ref{tab:wrms}. 
}
%we use the EQUAD command in {\sc tempo} to increase the errors on
%the TOAs by adding an extra amount of error to them in quadrature, such that
%the post-fit reduced $\chi^2$ is close to unity for the TOAs of every instrument. The amount
%of extra error added for different instruments are listed in Table
%\ref{tab:equad}.

%We use the DE421 solar ephemeris instead of the DE405 used by
%\citet{sns+05} for its improved precision on the masses of the Solar system
%planets, despite that using DE405 gave us marginally better $\chi^2$
%($\Delta\chi^2\sim20$ for 16750 degrees of freedom).
We used the JPL DE421 solar system ephemeris \citep{fwb09} to remove pulse
time-of-flight variation within the solar system. This ephemeris is oriented
to the International Celestial Reference Frame (ICRF) and thus our astrometric
results are also given in the ICRF frame. We note parenthetically that a
previous generation ephemeris, DE405, gave nearly identical, but marginally
better timing fits ($\Delta\chi^2\sim20$ for 16750 degrees of freedom)



The timing parameters and uncertainties are marginalized using a general least
square approach and the {\sc tempo} software package. 
We confirm that majority of the pulsar-binary parameters reported by
\citet{sns+05} also successfully describe the 21-year data set, except for the
high order frequency polynomials they used to treat timing noises. We also
detected for the first time an apparent decay of the binary orbit due to
relativistic effects.
The new timing model parameters (Table \ref{tab:par}) changed slightly from
those in \citet{sns+05} but consistent with their reported uncertainties.
%The parameters uncertainties are estimated from a Markov Chain Monte Carlo
%\footnote{{\bfref We used a Metropolis algorithm assuming the goodness of fit of
%the timing model is proportional to $e^{-\chi^2/2}$. The procedure is similar
%to that in the {\sc tempo2} MCMC plugin \citep{dpr+10}. The code for running this
%MCMC can be found here: \url{https://github.com/zhuww/tempomcmc}}}  
%(MCMC; \citealt{NR3}) simulation to determine the parameter uncertainties (Table \ref{tab:par}).


%\begin{itemize}
%\item T2 model versus DD? confirms that both model gives pretty much consistent results?
%\end{itemize}

\subsection{Noise model}
\label{sec:noise}
{\bfref 
There are two kinds of noises affecting our TOA measurements, uncorrelated
and correlated noise. Uncorrelated noises are 
independent from one TOA to another, while the correlated noises are not. 
For instance, the template matching errors and the radiometer noises are 
uncorrelated, the pulse jitter noise, which affects all the TOAs 
in the same epoch simultaneously, is epoch-correlated noise.
There are also time-correlated noises, such as red timing noises that are
correlated from epochs to epochs. 
The template matching error $\sigma$ can be estimated when we compute the
TOAs, and therefore, often come as the by-products of TOA measurements. 
Correctly accounting for these noises is a crucial for accurately
model the pulsar's timing behavior.
It has been proposed that the correlated and uncorrelated noises can be
modeled as Gaussian processes in a Bayesian approach. This approach has 
been discussed extensively in \citet{vl13,
ell13, vv14a, vv14, abb+14} and J. A. Ellis (PhD thesis 2014). In this
section, we outline the noise modeling procedure used in this paper. 


After we removed the timing model, the post-fit 
residuals $\msR$ should contain mostly Gaussian noises. Thus the 
likelihood function for given noise model is:
\begin{equation}
\label{eq:bigC}
p(\msR|\vec \phi) = \frac{1}{\sqrt{|2 \pi \msC|}}
\exp\left(-\frac{1}{2}\msR^T \msC^{-1} \msR \right),
\end{equation}
here $\vec \phi$ are the noise parameters and $\msC(\vec \phi)$ is the noise
covariance matrix. 
Assuming that our initial timing model is close to the true model, and their
difference is small enough to be regarded as a linear model, $\msR=\delta t
 - M \epsilon$, here $\delta t$ is the pre-fit residual,
$M$ is the design matrix of the timing model and $\epsilon$ denote small
offsets in the timing parameters. 

The noise covariance matrix $\msC$ could be used to describe both
correlated and uncorrelated noises. If we only have uncorrelated noises in our
post-fit residuals, then $\msC$ would be a diagonal matrix $\msD$. Conversely, correlated
noises would give rise to non-zero non-diagonal elements. $\msC_{ij}(i\neq
j)\neq 0$ indicates that $\msR_{i}$ is correlated with $\msR_{j}$. In the case
of pulse jitter noise, only the TOAs from the same sub-integration are correlated
with each other, thus $\msC$ would be a block diagonal matrix $\msJ$. If both types
of noise present, then $\msC = \msD + \msJ$.

In the past, extra noises are often modeled crudely using the timing software
packages {\sc tempo} and {\sc tempo2} by grouping TOAs according to their backends 
and receiver and assign each backend-receiver combination a EFAC and a EQUAD
parameter. Timing software then enlarge the uncertainty ($\sigma$) of 
the TOAs in the same group by a factor of EFAC ($E$),
and increase $\sigma$ by adding the value of EQUAD ($Q$) to it in quadrature.

We continue to use the EFAC and EQUAD parameterization for modeling 
uncorrelated noise like radiometer noise. The resulting diagonal covariance
matrix $D$ can be constructed with $\msD_{ii} = E_j^2\sigma_i^2 + Q_j^2$, where
$D_{ii}$ is the $i$th diagonal element of $\msD$, $E_j$ and $Q_j$ is the
EFAC and EQUAD values of the TOA group that the $i$th TOA belongs to. 

To model the correlated jitter noise, we employ another noise
parameter ECORR and parameterize the correlated
noise covariance matrix $\msJ_{ij} = J_k^2$, where TOA $i$ and $j$ come from
the same sub-integration, and $J_k$ is the ECORR value for the $k$th backend-receiver group.
We can further simplify the matrix $\msJ$ by rewriting it as 
$\msJ = U \tilde{\msJ} U^{T}$, where $\tilde{\msJ}$ is a
N$_{\rm subint}\times$N$_{\rm subint}$ diagonal matrix with $\tilde{\msJ}_{jj} =
J_k$ when the $j$th TOA sub-integration belongs to the $k$th backend-receiver
group; and $U$ is a N$_{\rm TOA}\times$N$_{\rm subint}$ projection matrix,
$U_{ij} = 1$ when the $i$th TOA comes from the $j$th sub-integration, otherwise $U_{ij}=0$. 

As a result, the noise covariance matrix constructed from the EFAC, EQUAD, and
ECORR parameters is:
\begin{equation}
\label{eq:decom}
\msC = \msD + U \tilde{\msJ} U^{T}.
\end{equation}
To maximize the likelihood $p(\msR|\vec \phi)$ requires inverting the large covariance 
matrix $\msC$ many times. A direct approach is too computationally costly to be viable, we had to employ some linear algebra tricks to make this work. 


It was shown by \citet{vl13} that likelihood function in Equation
\ref{eq:bigC} can be marginalized over timing parameters and re-written as:
\begin{equation}
\label{eq:Gintro}
p(\delta t|\vec \phi) = \frac{\exp(-\frac{1}{2}\delta t^TG(G^T \msC G)^{-1}G^T\delta t)}
{\sqrt{(2\pi)^{n-m}|G^{T}\msC G|}},
\end{equation}
where $n$ denote the total number of TOAs, and $m$ denote the total number of
timing parameters.
Here the matrix $G$ comes from the singular variable decomposition of the design
matrix: $M = U\Sigma V^*$, where U and V are $n\times n$ and $m\times m$ orthogonal
matrix, and $\Sigma$ is a ($n\times m$) diagonal matrix.
We define $U = (F G)$, where $F$ consists of the first $m$ columns of $U$ and
$G$ consists of the next ($n-m$) columns of $U$.

Using equation \ref{eq:decom} and the Woodbury Lemma
\footnote{$(A+DBE^T)^{-1}=A^{-1}-A^{-1}D(B^{-1}+E^TA^{-1}D)^{-1}E^TA^{-1}$ and
$|A+DBE^T| = |A||B||B^{-1}+E^TA^{-1}D|$}, we get (equation 25 of
\citealt{abb+14}):
\begin{equation}
\label{eq:Gform}
p(\delta t|\vec \phi) = \frac{\exp[-\frac{1}{2}(\delta t^T \tilde{\msD}^{-1} \delta t -
d^T\msS^{-1} d)]}{\sqrt{(2\pi)^{(n-m)}|\tilde{\msJ}||G^T \msD
G||\msS|}},
\end{equation}
where $\tilde{\msD}^{-1} = G(G^T \msD G)^{-1}G^T$, $d=U^T \tilde{\msD}^{-1}\delta
t$, and $\msS = (\tilde{\msJ}^{-1} + U^T \tilde{\msD}^{-1}U)$.
Here $\tilde{\msD}^{-1}$ can also be calculated using the $F$ matrix 
(see equation 29 in \citealt{vv14a}):
\begin{equation}
\label{eq:tildeD}
\tilde{\msD}^{-1} = G(G^T\msD G)^{-1}G^T = \msD^{-1} -
\msD^{-1}F(F^T\msD^{-1}F)^{-1}F^T\msD^{-1}.
\end{equation}

The $F$ formalism of $\tilde{\msD}^{-1}$ is much easier to compute because $F^T\msD^{-1}F$ is a $m\times m$ matrix whereas $G^T\msD G$ is $(n-m)\times(n-m)$.
Similarly, the determinants in the denominator also can be computed using a $F$ matrix formalism: $|G^{T}\msD G| = |\msD||F^T\msD^{-1}F|$ (See appendix A of \citet{vv14a} for a more detailed proof).
This is because orthogonal transformations like $U$ keep determinants invariant, and 
\begin{equation}
|\msD| = |U^T \msD U| = |G^T\msD G||(F^T\msD^{-1}F)^{-1}| = |G^T\msD G|/|F^T
\msD^{-1} F|.
\end{equation}
After using the $F$ matrix to reduce matrix ranks, the matrix inversion
$F^T\msD^{-1}F$ and its determinant can be computed quickly using Cholesky
factorization:
$LL^* = F^T\msD^{-1}F$, where $L$ is the lower(or upper) triangle matrix, and $|F^T\msD^{-1}F| = |L|^2 = (\prod_i L_{ii})^2$.

With the $F$ matrix formalism, $p(\delta t|\vec \phi)$ can be computed
efficiently. 
We wrote a program\footnote{\url{https://github.com/zhuww/noisemodel}} to search over many combinations of noise parameters $\vec
\phi$ (EFAC, EQUAD, and ECORR) to maximize the likelihood
function. This program works in conjunction with the {\sc tempo} software package.
We use {\sc tempo}'s general least square fit routine to fit for the
timing model and our code to model the noise parameters, and run both programs
in alternation for few iterations to make sure that both the timing model and the
noise model are marginalized simultaneously.

In our noise model, we marginalized over the EFAC, EQUAD, and ECORR parameters
for data collected by different backend and receivers systems. However, we did not model 
EFAC and ECORR of the Mark~III, Mark~IV, and ABPP data. Because there were not
enough number of TOAs in the legacy dataset to constrain both EFAC and EQUAD,
there were only one TOA per epoch so we cannot constrain the epoch-correlated
noise modeled by ECORR such as jitter noise. Instead, we set EFAC values to 1
and ECORR to 0 for these data sets, and use only EQUAD to model extra noises
in them.

%\subsection{Timing residuals}
%\label{sec:res}

The timing residuals of PSR~J1713+0747 have a weighted root mean square (WRMS)
$\sim 245$ns.
If we average the residuals in each epoch, the epoch-averaged residuals has a 
WRMS of $\sim 118$ns (Figure \ref{fig:res}, Table \ref{tab:wrms}).
Here we define the error weighted epoch-averaged residual as: 
$\msR_{\rm epoch} = \langle w_j\msR_{j}\rangle$, $\sigma_{\rm
epoch}=\{\sum [(\msR_j-\msR_{\rm epoch})^2w_j]/(N-1)/\sum w_j\}^{1/2}$, where $j$
denote one of the $N$ TOAs from the respective observation epoch and
$w_j=1/\sigma_j^2$.

Our noise and residual analysis shows that the advances in
instrumentation have made pulsar timing increasingly accurate.
The average timing residuals from ASP, GASP, GUPPI and PUPPI, are
systematically smaller than those from the earlier Mark~III, Mark~IV and ABPP
systems.

\citet{sc12} studied the pulse arrival times from a single long exposure of
PSR~J1713+0747, and found that this pulsar's single pulse showed random jitter of
$\simeq26~\mu$s. Therefore, by averaging many pulses collected in the
$\sim20$~min NANOGrav observation, one expect $\sim26\mu{\rm s}/\sqrt{1200\nu}=50$~ns of jitter noise. 
The optimal jitter parameters (ECORR, as shown in Table \ref{tab:wrms}) from
our noise modeling are mostly consistent with the prediction from
\citet{sc12}, with some of them being higher. This could be due to the
covariance between the jitter parameters and the EQUAD parameters.
%It is reassuring that the predicted level jitter noise is about the same order
%of magnitude as our ECORR values and the smallest residual RMS (Table \ref{tab:wrms}). 

}

%Fitting for DM variations on short timescales is necessitated by the sharp
%variations seen in DM (Figure \ref{fig:dmx}). However, this has the potential
%effect of overfitting the model and artificially reducing the spread in the
%residuals.
%This is especially significant for the lower band residuals, because
%the DM fitting affects the lower frequency TOAs more than the
%higher frequency TOAs, and is more likely to flatten their timing residuals. 
%As a result we expect the residuals' WRMS to be 
%underestimated. This effect can be crudely quantified
%by injecting random Gaussian noise to the TOAs and
%measuring the response in the post-fit residuals. 
%We perform this simulation one system at a time. 
%We added Gaussian noise to the TOAs of each observation epoch of the giving system.
%The noise are random between different epochs, but they are the
%same for all the TOAs of different frequencies in any given epoch. 
%Figure \ref{fig:overfit} shows
%the WRMS of the post-fit residuals as a function of injected
%noise level for the PUPPI 1400-MHz TOAs. 
%One can see that the RMS residual increases almost linearly as a function of the
%injected noise, but only a fraction of the injected noise is recovered in the timing 
%residual. Based on such simulations, we
%can estimate the level of noises ``absorbed'' by the timing models, and infer 
%the ``corrected amplitude'' of the timing residuals. 
%Table \ref{tab:wrms} shows the WRMS measured directly from the residuals 
%and the ``corrected RMS'' based on noise injection simulations. 
%These ``corrected RMSs'' serve to demonstrate the effect of ``overfitting''.
%They are not the proper measurements of noise level in the data. There are some 
%caveats in our approach, firstly, we only simulate noises between different
%epochs, not noises between different frequencies of the same epoch; secondly, 
%the injected noise is Gaussian white noise, while the actual 
%noise may contain some red and non-Gaussian noises. 






\subsection{Mass measurements}
\label{sec:mass}
The timing model of PSR~J1713+0747 has been significantly improved by the 21-year timing effort.
Most notably, the pulsar and the companion masses have been more precisely
constrained (Table \ref{tab:par}) through Shapiro delay measurements. The
companion's mass $M_{\rm c} = 0.290\pm0.014$~\Msun and the pulsar $M_{\rm
PSR}=1.3\pm0.1$~\Msun are in good agreement with the previously measured values \cite{sns+05}.
We note that the derived pulsar mass tend to be underestimated 
if no jitter parameters were included in our noise model, suggesting that
correlated noises could significantly impact the result of high
precision timing analysis.


The pulsar's mass is comparable with the distribution of pulsar masses
in other neutron star-white dwarf systems, and in good
agreement with the distribution of pulsar masses found in recycled binaries
\citep{opns12,kkdt13}. The precise measurement of neutron star mass like this
may eventually help us understand the properties of matter in extreme 
density \citep{lat12}.

In the standard picture of binary evolution, a MSP with a low-mass white dwarf companion must have been spun up through accretion when the white dwarf was a giant star filling its Roche lobe. 
This should lead to a strong correlation between the binary period and the mass of the white dwarf companion \citep{rpj+95, ts99a, prp02b}. 
Indeed, this picture has been supported by the measurements of several pulsar
binary systems \citep[e.g.,][]{vbb+01, ktr94, th14}.  
The orbital period and companion mass of PSR~J1713+0747 fits
this correlation very well, thus supporting the standard MSP evolution theory. %Notably, not all MSP companions follow the same orbital period -- companion
%mass relation, for instance, J1903+0327 has an orbit of 95~days but a
%companion of $\sim 1M_{\odot}$ \citep{fbw+11}. Such peculiar systems requires
%more than the simple scenario considered above to explain.

%Some early statistical analysis shows that pulsars 
%might have a very narrow mass distribution 1.35$\pm0.04M_{\odot}$ \citep{tc99}. 
%However, several massive ($\sim2M_{\odot}$) pulsars have been found in recent
%years (J1614$-$2230 1.97$\pm0.04M_{\odot}$\citealt{dpr+10}; J1913+0327
%1.67$\pm0.02M_{\odot}$ \citealt{fbw+11}, J0348+0432 2.01$\pm0.04M_{\odot}$
%\citealt{afw+13}) that clearly out-lie this distribution.

%J1713+0747 appears to be nominal when compared with the early year samples
%despite of its recycling history.


\subsection{DM variation}
\label{sec:dmx}
The DM of a pulsar reflects the amount of free electrons between
the pulsar and the telescopes and it varies because
our sight-line through the turbulent ISM and solar wind is changing as the
pulsar, the Sun, the Earth, and the ISM all move with respect to each other.
The DM variation can affect the timing of high-precision pulsar significantly.

{\bfref
We fit simultaneously with other parameters the time-varying DM using the {\it DMX} model in {\sc tempo}.
This model fits independent DM values for TOAs groups taken within 14 days
intervals, except for the L-band-only Mark~III TOAs. We grouped the Mark~III
TOAs together as a single group, because their frequency resolution and timing
precision are not sufficient for measuring epoch-to-epoch DM changes.

%The best-fit {\it DMX} model is consist of a DM time series over the length 
%of the data set.
%We made sure that in each group we have TOAs
%from at least two different bands (such as L-band and S-band). 
%The same procedure was carried out in
%\citet{dfg+13}.  This DM fitting was conducted with TOAs from all
%instruments except Mark~III, which produced only single-frequency TOAs.

%The grouping is feasible because of the frequency of our multi-band
%observations.
%The Arecibo 1400-MHz and 2300-MHz observations of PSR~J1713+0747 almost
%always happen consecutively on
%the same day, while the GBT 800-MHz and 1400-MHz observations often take place 
%on different days spaced by only few days apart.
%Therefore, the DM values in our {\it DMX} model were determined by TOAs from
%similar epochs and from at least two frequency bands. 
Figure \ref{fig:dmx} shows the measured DM variation of PSR~J1713+0747.
Note that the systematic drop of DM around MJD 53200 is likely instrumental, due to
the use of different standard pulse profile templates when extracting 
TOAs from the Mark~IV, ABPP data. The DM values inferred from different
profile templates are expected to have a constant offset.
Conversely, the sudden dip and recovery of DM around MJD 54800 is 
due to changes either in the ISM or in the solar wind. This DM dip is also
observed independently by the Parkes observatory \citep{kcs+13}.
There is clear red-noise-like variance in the pulsar's DM with a RMS of
$\sim10^{-4}$. 
%This DM variation may lead to ``red'' timing residuals as discussed in \citealt{kcs+13}.
}

%Note that the first part of this DM curve (MJD$<53200$) in
%Figure \ref{fig:dmx} has much larger uncertainties than the second half.
%This is because the relative inferior bandwidth of the Mark~IV and ABPP
%data. The Mark~IV data in particular have much smaller frequency bandwidth 
%compared to the later data (Table \ref{tab:obs}).
The ABPP TOAs we have are epoch-averaged TOAs, meaning they are collapsed in frequency to form one TOA per band per observation epoch, thus losing frequency resolution in the respective band. 
This averaging was done using an previous timing solution in previous work
(\citep{sns+05}). Thus the ABPP TOAs may be slighted biased to favor the
timing parameters in the previous solution. We also included the original
before-epoch-average Mark~IV TOAs, which were taken in the same epochs of the
ABPP data. Both of the Mark~IV and ABPP  TOAs are well fitted by our new
timing solution.
%This makes it hard to resolve the profile evolution in
%frequency that we see in the later data. When summing the data in frequency, we
%also assume the center frequency of the band is the middle frequency, while the effective center frequency may be different due to pulse profile evolution and/or scintillation. 
%This procedure is likely to produce TOAs that are slightly offset. We think this may be why the FD and
%DMX models cannot completely model the red-noise signal in these TOAs, leaving
%an apparent hump in the first part of the DM curve (Figure \ref{fig:dmx}) and 
%a similar shape in the 2300-MHz timing residuals from Mark~IV and ABPP (Figure
%\ref{fig:res}).

%There is a hint of small annual DM variations, part of which may be 
%related to the varying electron density in the Solar wind as our sight line moves.
%The Earth's motion could also cause a
%helical motion through the ISM that leads to an annual DM variation.

Spectrum analysis of the time variation of flux, pulse arrival phase, and DM have 
been employed to study the turbulent nature of the ISM since \citealt{cpl86, rl90}.
It has been shown that the DM variation of some pulsars are consistent with
those expected from an ISM characterized by a Kolmogorov turbulence spectrum
\citep{cwd+90, ric90, ktr94, yhc+07, kcs+13, fst14}. In those cases one can calculate the 
structure function of the varying DM: 
\begin{equation}
D_{\phi}(\tau)=\left(\frac{2\pi K}{f^2}\right)\langle [DM(t+\tau)-DM(t)]^2\rangle, 
\end{equation}
where $\tau$ 
is a given time delay, $K=4.148\times10^3$~MHz$^2$pc$^{-1}$cm$^3$s, and $f$ is 
the observing frequency in MHz, and expect 
this function to follow a Kolmogorov power law $D_{\phi}(\tau)=(\tau/\tau_0)^{\beta -2}$, 
where $\beta=11/3$ and $\tau_0$ is a characteristic time scale related to 
the inner scale of the turbulence. The pulsars with DM variation that fits this
theory generally have large DM variations in timescale of 
years. However, the DM variation of PSR J1713+0747 (Figure \ref{fig:dmx}) does not 
follow the same characterization. It went trough a steep drop and recovery 
around 2008. Conversely, the overall long-term variation is smaller
compared with the 2008 event. As a result, its structure function (Figure \ref{fig:dmx}) 
follows a flatter spectrum than the Kolmogorov one.


\subsection{Pulsar spin irregularity}
\label{sec:noise}
%The pulsar timing residuals (Figure \ref{fig:res}) are the residuals left
%after we fit the TOAs with our timing model. 
%These residuals are likely caused by timing noise, DM variations, radiometer 
%noise, and other sources of noise. This section discusses PSR~J1713+0747's
%timing noise, i.e. TOA variations due to spin irregularities.

The term ``timing noise'' in pulsar timing generally refers to the non-white
noises left in the residuals of timing modeling.
An important part of these timing noise are expected to come from pulsar's spin
irregularity, i.e. its long-term deviation from a simple linear slow down. 
Spin irregularity is often significant in younger pulsars, and often
requires modeling with significant high-order frequency polynomials (such as $\ddot{\nu}$, where $\nu$ is the pulsar's spin frequency). 
Potential causes of irregular spin behavior include unresolved
micro-glitches, internal superfluid turbulence, magnetosphere variations, or external torques caused by matter surrounding the pulsar \citep{hlk10, ymh+13, ml14}.
{\bfref These mechanisms could lead to accumulative random perturbations in the 
pulsar's phase of pulse arrival, spin rate, or spin-down rate. 
\citet{sc10} point out that one could model these types of timing noises using random walks.
Random walks in phase (RW$_0$) would grow over time ($T$) proportional to
$T^{1/2}$, random walk in $\nu$ grows proportional to $T^{3/2}$, random walk in
$\dot{\nu}$ grows proportional to $T^{5/2}$.
Such spin noises would likely have a steep power spectrum with more power in
the lower frequencies, also known as a ``red'' power spectrum. They
are considered as one of the main sources of ``red'' noises in pulsar timing.
}

The timing noise of radio pulsars has been studied by
\citet{ch80,cd85,antt94,dmhd95, mtem97}, and later by \citet{hlk10} and
\citet{sc10} with large samples. 
%\citet{antt94} characterized the significance of timing noise using the second
%spin derivative $\ddot{\nu}$. 
%This is because for most pulsars except the few youngest ones, 
%The expected $\ddot{\nu}$ from regular spin down is too small to be
%measurable, therefore the higher order spin parameters we observe are most
%likely the result of timing noise.
%They define the noise factor 
%\begin{equation}
%\label{eq:delta8}
%\Delta_8 = \log_{10}\left(\frac{1}{5\nu}|\ddot{\nu}|t^3\right).
%\end{equation}
%Here $t=10^8$~s$\sim 3$~yr is a fiducial time scale close to their average
%observation time span.
%Similarly, 
\citet{mtem97} adopted a generalized Allen Variance (traditionally used in
measuring clock stability) to characterize the timing instability of pulsars:
\begin{equation}
\label{eq:sigmaz}
\sigma_z(\tau) = \frac{\tau^2}{2\sqrt{5}}\langle c^2 \rangle^{1/2},
\end{equation}
where $\langle c^2\rangle$ denote the sum of squares of the cubic
terms fitted to segments of length $\tau$). 
$\Delta_8$ is the logarithm of a dimensionless form of $\sigma_z(10^8~s)$.
{\bfref
\citet{hlk10} found a best-fit scaling model of $\sigma_z({\rm 10~yr})$ 
from 366 pulsars:
\begin{equation}
\label{eq:hlk10}
\log_{10}[\sigma_z({\rm 10~yr})] =
-1.37\log_{10}[\nu^{0.29}|\dot{\nu}|^{0.55}]+0.52,
\end{equation} 
where $\nu$, $\dot{\nu}$ are the pulsar's spin and spin-down rate.
We find that \citet{hlk10}'s scaling model ($\sigma^{\rm model}_{z, \rm
10~yr}\simeq1\times10^{-12}$) over-predict $\sigma^{\rm measured}_{z, \rm
10~yr}=1.5\times10^{-15}$ by more than two orders of magnitude. 
%indicating that PSR~J1713+0747 has
%significantly less timing noise than average pulsars of similar spin behavior.

\citet{ch80} defined a different timing noise characteristic $\sigma^2_{\rm
TN,2}$ based on the root mean square of residuals $\sigma^2_{\msR,2}$ from a
timing fit that does not include any higher order spin parameters like
$\ddot{\nu}$. 
The timing noise term is related to $\sigma^2_{\msR,2}$:
\begin{equation}
\sigma^2_{\msR,2}(T) = \sigma^2_{\rm TN,2}(T) + \sigma^2_W, 
\end{equation}
where $\sigma^2_W$ is a time-scale $T$ independent term caused by white 
noise in the data.
In this definition, timing noise $\sigma^2_{\rm TN,2}(T)$ grows bigger over
time while white noise stays constant.  

\citet{sc10} studied the $\sigma^2_{\rm TN,2}$ from a large sample of pulsars
including canonical pulsars (CPs) and millisecond pulsars, they found a scaling
model:
\begin{equation}
\label{eq:sc10}
\ln(\hat{\sigma}_{\rm TN,2}) = 1.6 - 1.4\ln(\nu) +
1.1\ln|\dot{\nu}_{-15}|+2\ln(T_{\rm yr}),
\end{equation}
where $\dot{\nu}_{-15}$ is $\dot{\nu}$ in units of $10^{-15}$s$^{-2}$, $T_{\rm yr}$
is the observation time span in years.
This scaling model predict that, for 21-year timing of PSR~J1713+0747, the
residual RMS without removing timing noises $\sigma^2_{\rm TN,2}$ would be
$\sim0.4$~$\mu $s. The measured RMS of the red noise residual 
$\sigma^2_{\msR,RN}=395$~ns, is consistent with the extrapolation
from \citet{sc10}. 

%Figure \ref{fig:TN} shows $\sigma^2_{\msR,2}$ for different time scales, and
%there is no significant evidence of a growing timing noise. We compared 
%$\sigma^2_{\msR,2}(T)$ with a white noise model as well as different random
%walk models (RW$_0$, RW$_1$, RW$_2$) to show that our timing residual is
%mostly white.
%This result is also consistent with the timing noise analysis of this pulsar in 
%\citet{pjl+13}. They studied the timing noise using an auto-correlation technique
%based on 5 year NANOGrav data, and conclude that the timing residuals are
%consistent with mostly white noise.
}


\subsection{Pulse profile evolution in frequency}
\label{sec:FD}
After removing the dispersion that causes TOA delays of $\propto \nu^{-2}$,
 we still see small remaining frequency-dependent residuals from wide-band
observations using
different instruments and different telescopes (Figure \ref{fig:FD}).  
The cause of such profile evolution is likely a change in the pulsar's
radiation pattern with frequency.  Pulsar radiation of different frequencies may originate from
different parts of the star's magnetosphere, and 
the radiation region of the pulsars' magnetosphere may be slightly distorted,
leading to a frequency-dependent radiation pattern. \citet{pdr14} 
extensively discussed this phenomenon and developed a TOA extraction technique
based on phase-frequency 2-D pulse profiles matching. This technique is not
yet applied to our dataset.

\citet{sns+05} allowed an arbitrary offset between TOAs taken with different
observing systems and at different frequencies.
However, the number of frequency
channels has increased by a factor of ten with the modern wide-band
instruments, making it a lot harder to mitigate profile-frequency evolution using jumps. 
Instead, we used the {\it FD} model, a polynomial of the logarithm of
frequency (Demorest et~al.\ in prep.; solid line in Figure
\ref{fig:FD}) to fit for and removed the profile-frequency
evolution. This model successfully removes the extra
frequency-dependent residuals and it requires only four parameters in the
case of PSR~J1713+0747.





\section{Results}
\label{sec:res}

\subsection{Intrinsic orbital decay}
\label{sec:obdecay}
We have measured an orbital decay from PSR~J1713+0747, $\dot{P}_{\rm b} =
0.48\pm0.18\times10^{-12}$s~s$^{-1}$ (Table \ref{tab:par}).
%Such a precise measurement can be used to test Gravitational wave radiation theories of Einstein's and beyond.
This orbital decay may not be intrinsic to the pulsar binary, but rather the
result of the kinetic motion between the binary and the
observer, i.e. a relativistic effect caused by differential 
acceleration in the Galactic gravitational potential
\citep{dt91} and a relativistic effect caused by the proper motion of the
pulsar. The relativistic effect
of proper motion is also known as the ``Shklovskii'' effect (
\citealt{shk70}). Luckily, we have good measurements of the distance and proper
motion of the binary system, which allow us to remove these effects and study the system's intrinsic orbital decay.
\begin{equation}
\dot{P}_{\rm b}^{\rm Gal} = \frac{A_{\rm G}}{c} P_{\rm b} =
-0.09\pm0.02\times10^{-12}~{\rm s~s^{-1}}
\end{equation}
where $A_{\rm G}$ is the line-of-sight acceleration of the pulsar binary;
this term is dominated by the difference in the Galactic accelerations of the
binary and our solar system, and is obtained using
Equation 5 in \citet{nt95}, Equation 17 in \citet{lwj+09} and the Galactic
potential model by \citet{hf04a}.
On the other hand, the Shklovskii effect causes $P_{\rm b}$ to
change by
\begin{equation}
\dot{P}_{\rm b}^{\rm Shk} = (\mu_{\alpha}^2+\mu_{\delta}^2)\frac{d}{c}P_{\rm
b} = 0.66\pm0.03\times10^{-12}~{\rm s~s^{-1}}.
\end{equation}
Therefore, the pulsar's intrinsic orbital decay is $\dot{P}_{\rm b}^{\rm Int}
= \dot{P}_{\rm b}^{\rm Obs} - \dot{P}_{\rm b}^{\rm Shk} - \dot{P}_{\rm b}^{\rm
Gal} = (-0.09\pm0.20)\times10^{-12}$s~s$^{-1}$, and is consistent with zero.

Due to the very long $\sim$68 day orbit, the binary's decay due to the
emission of gravitational
radiation is expected to be undetectable: $\dot{P}_{\rm b}^{\rm GR} =
-6\times10^{-18}$s~s$^{-1}$ \citep{lk05}.  Therefore, the unmeasurable small 
intrinsic orbital decay rate is entirely consistent with the
description of quadrupolar gravitational radiation within General
Relativity (GR).

Other than the gravitational radiation, there are two other effects could have played a role in
$\dot{P}_{\rm b}^{\rm Int}$. One, $\dot{P}_{\rm b}^{\dot{M}}$, is caused by mass loss in the
binary system, and the other, $\dot{P}_{\rm b}^{\rm T}$, is the contribution
from tidal effects.
The pulsar and the white dwarf both could lose mass due to their magnetic dipole radiation; the maximum
mass loss rate due to this effect can be estimated from loss rate of the
star's rotational energy. In the case of the pulsar, $\dot{M_{\rm
PSR}}=\dot{E}/c^2$, that is
measurable through the spin down rate of the pulsar.
The white dwarf generally loses mass at a much lower rate than the pulsar.
%\begin{equation}
%\dot{P}_{\rm b}^{\dot{m}} = 8\pi^2\frac{I_{\rm
%PSR}}{Mc^2}\frac{\dot{P}}{P^3}P_{\rm b} \sim 10^{-16},
%\end{equation}
%where $M=M_{\rm PSR} +M_{\rm WD}$ is the total mass of the system and
%$I\sim10^{45}$g~cm$^2$ is the angular momentum of inertia.
Therefore, orbital change due to mass loss can be estimated as $\dot{P}_{\rm
b}^{\dot{M}}\sim 1\times10^{-14}$s~s$^{-1}$ (\citealt{dt91}; Equation 9 and 10
of \citealt{fwe+12}). This is an order of magnitude smaller than the measured
uncertainties on $\dot{P}_{\rm b}^{\rm Int}$.
The tidal effect in this binary system is expected to be $\dot{P}_{\rm b}^{\rm
T}\ll1\times10^{-14}$s~s$^{-1}$ based on the most extreme scenarios (the white
dwarf spins at its break-up velocity and the tidal synchronizing time scale equals the
characteristic age of the pulsar; see Equation 11 in \citealt{fwe+12} and
references therein in).
Both of these extra terms are much smaller than the observed uncertainties
on $\dot{P}_{\rm b}^{\rm Int}$.
%The tidal effect can be estimated to $\sim$ according to Equation 11 of
%\citealt{fwe+12} and references therein. 


\subsection{Time Variation of $G$}
\label{sec:Gdot}


%The wide orbit of PSR J1713+0747 ($\sim$68 days) makes it 
%some alternative theories of gravity, such as scalar-tensor gravity,
%predict larger orbital decays.
%Consequently, measurements of pulsar binaries' ``excess'' orbital decay
%$\dot{P}_{\rm b}^{\rm exc}=\dot{P}_{\rm b}^{\rm Int} - \dot{P}_{\rm
%b}^{\dot{M}}  - \dot{P}_{\rm b}^{\rm T} - \dot{P}_{\rm b}^{\rm GR}$ have been
%used to constrain alternative theories \citep[e.g.][]{lwj+09, fwe+12}. 
Based on the measurement of the ``excess'' orbital decay 
$\dot{P}_{\rm b}^{\rm exc}=\dot{P}_{\rm b}^{\rm Int} - \dot{P}_{\rm
b}^{\dot{M}}  - \dot{P}_{\rm b}^{\rm T} - \dot{P}_{\rm b}^{\rm GR}$,
\citet{dgt88} derived a generic phenomenological limit for $\dot{G}$: 
$\dot{G}/G\simeq-\dot{P}_{\rm b}^{\rm exc}/(2P_{\rm
b})=1.0\pm2.3\times10^{-11}$~yr$^{-1}$ using timing of binary PSR~1913+16. 
{\bfref Since then $\dot{P}_{\rm b}^{\rm exc}$ of pulsar binaries, including 
PSR~J1713+0747, have been used to 
constrain $\dot{G}/G$ \citep{ktr94, lwj+09, fwe+12, nss+05}. }
So far all pulsar observations show $\dot{G}/G$ consistent with being zero, with 
upper limits largely determined by the uncertainties in $\dot P_{\rm b}$, distance, 
and proper motions.
%Thanks to the effort of NANOGrav timing project, all these parameters are 
%measured for PSR~J1713+0747 to great precision.  
PSR~J1713+0747 has the smallest known $\dot{P}_{\rm b}^{\rm exc}/(2P_{\rm
b})\simeq(-0.2\pm1)\times10^{-12}$~yr$^{-1}$ (Section \ref{sec:obdecay}) and is
particularly useful for constraining the time variability of gravitational
constant .

However, we would like to approach the $\dot{G}$ constrain in a more rigorous fashion,
by incorporating a class of realistic alternative gravitational theory --- the 
scalar-tensor theory.
In the framework of this theory,
the scalar field that interacts with the mass changes over time as the
Universe expands. This change will cause the local value of
$G$ to vary. The changing $G$ will also change
the orbit of a binary system:
\begin{equation}
\dot{P}_{\rm b}^{\dot{G}} = -2 \frac{\dot{G}}{G}
\left[1-\left( 1+\frac{m_c}{2M}\right) s_p\right]P_{\rm b},
\end{equation} \citep{nor90}.
This formalism is only slightly different from the generic form of
\citet{dgt88}.

Meanwhile, in the framework of an alternative gravitation theory that violates
SEP, a binary system may emit dipole gravitational radiation \citealt{Will93, Will01, lwj+09, fwe+12} and references
therein). Such effects arise when the two bodies are very different in terms
of their self-gravity, i.e.  their compactness.
This extra dipole radiation could lead to an extra orbital decay term:
\begin{equation}
\dot{P}_{\rm b}^{\rm D} \simeq -4\pi\frac{T_{\odot}\mu}{P_{\rm b}}\kappa_D S^2,
\end{equation}
\citep{lwj+09}, where $T_{\odot}=G{\rm M_{\odot}}/c^3=4.925490947$~${\rm
\mu}$s \citep{lk05}, $\mu$ is the reduced mass $m_pm_c/M$ of the system , $\kappa_D $ is dipole
gravitational radiation ``coupling constant'', and $S$ is the difference
between the self-gravity ``sensitivity'' of the two bodies ($S = s_p - s_c$;
$s_p\sim0.1m_p/M_{\odot}$ according to \citealt{de92} ; and $s_c\ll s_p$).
In Einstein's general relativity $\kappa_D=0$ --- there is no self-gravity induced
dipole gravitational radiation, but it is often not the case in alternative
theories that violate the SEP.

PSR~J1713+0747 has a wider binary orbit than most other
high-timing-precision pulsar binaries, making its $\dot{P}_{\rm b}^{\rm D}$
very small. Conversely, $\dot{P}_{\rm b}^{\dot{G}}$ is larger when $P_{\rm b}$
is large. This makes PSR~J1713+0747 the best pulsar binary system for constraining
the effect of the changing gravitational constant $\dot{G}$. Limits 
on both $\dot{G}$ and $\kappa_D$ can be estimated in the same fashion as in
\citet{lwj+09}: by solving $\dot{G}$ and $\kappa_D$ simultaneously 
from the equation $\dot{P}_{\rm b}^{\rm exc} = \dot{P}_{\rm b}^{\rm D} +
\dot{P}_{\rm b}^{\dot{G}}$ (Equation 29 of \citealt{lwj+09}) of different
pulsars. We applied this method to four pulsars: PSR 0437$-$4715, PSR J1012+5307, PSR
J1738+0333, and PSR~J1713+0747 using timing parameters reported in
\citet{lwj+09}, \citet{fwe+12}, and this work.
The resulting confidence region of $\dot{G}$ and $\kappa_D$ is shown in Figure
\ref{fig:Gdot}.
We found, at 95\% confidence limit, $\dot{G}/G =
0.3\pm1.1\times10^{-12}$~yr$^{-1}$; $\kappa_D=-0.6\pm3.4\times10^{-4}$. 
This constraint on $\dot{G}$ is more than a factor of two more stringent than
previous pulsar-based constraints \citep{fwe+12}, and close to 
the best constraint of this type
($\dot{G}/G=-0.7\pm7.6\times10^{-13}$~yr$^{-1}$) from the Lunar Laser Ranging
(LLR)
experiment \citep{hmb10}, which measured Luna-Earth distance to $10^{-11}$
precision using 39 years of observations.
The pulsar-timing $\dot{G}$ and $\kappa_D$ limits are particularly interesting 
in the framework of the SEP-violating alternative theories, because they are from 
a test using objects of strongly self-gravitation.

\subsection{Strong equivalence principle and Lorentz invariance}
\label{sec:sep}
General relativity is the only gravitation theory that satisfies
the SEP, which states that the gravitational
effect on a small test body is independent of its constitution, specifically,
this principle holds that bodies of different self-gravitation should behave the same in
the same gravitational experiments. This principle is violated in alternative
theories of gravitations like the aforementioned Jordan-Brans-Dicke
scalar-tensor theory. The PSR~J1713+0747 binary is an excellent laboratory for testing 
effects of SEP violation. If the SEP is violated, the neutron star and the white
dwarf will be accelerated differently by the Galactic gravity field, causing
the binary orbit to be polarized toward the center of the Galaxy. The excess 
eccentricity is expected to be (\citealt{ds91}):
\begin{equation}
|\textbf{\textit{e}}_F| = \frac{1}{2}\frac{\Delta\, g_{\bot}
  c^2}{G(M_{\rm PSR}+M_{\rm
c})(2\pi/P_{\rm b})^2},
\end{equation}
where $g_{\bot}$ is the projection of Galactic acceleration on the orbital plane 
and $\Delta$ is the dimensionless factor that characterizes the significance 
of SEP violation. A recent model of the Galactic acceleration can be found in
\citet{hf04a}.

Lorentz invariance is another principle that is satisfied by GR and may
be violated by alternative theories. This principle
states that there is no preferred inertial reference frame. The violation of
Lorentz invariance would lead to polarization of binary orbits along the
direction of a preferred frame.
One can estimate such an effect in a strong-field version of the Post-Newtonian Parameterization
(PPN) framework\cite{de92}; the excess eccentricity is expected to be (\citealt{bd96}):
\begin{equation}
|\textbf{\textit{e}}_F| = \hat{\alpha}_3 \frac{c_p|\textbf{\textit{w}}|P_{\rm b}^2}{24\pi P}
\frac{c^2}{G (M_{PSR}+M_{\rm c})}\sin \beta,
\end{equation}
where $\textbf{\textit{w}}$ is the absolute velocity of the binary system
relative to the preferred frame of reference, typically taken as that of the cosmic microwave background (CMB), $P$ is the pulsar's spin period, $\beta$ is the
angle between $\textbf{\textit{w}}$ and the spin axis of the pulsar, and
$\hat{\alpha}_3$ is the strong-field version of one of the Post-Newtonian parameters that characterize the violation of Lorentz invariance. 
Here $\textbf{\textit{w}} = \textbf{\textit{w}}_{\odot} + \textbf{\textit{v}}_{\rm PSR}$, where
$\textbf{\textit{w}}_{\odot}=384\pm139$km~s$^{-1}$ is the velocity of
the solar system relative to the CMB (\citealt{aaa+13}),
and $\textbf{\textit{v}}_{\rm PSR}$ is the relative speed of the pulsar to our solar system. $\textbf{\textit{v}}_{\rm PSR}$ is only partially known because we can measure the pulsar's
proper motion on the sky but we cannot measure its radial velocity.

Fortunately, many variables in these equations are measurable in the
case of the PSR~J1713+0747 binary. This makes it possible to constrain $\Delta$
and $\hat{\alpha}_3$ using Bayesian techniques 
%assuming certain fiducial priors for the unmeasurable variables like the radial component of
%$v_{\rm PSR}$ 
\citep{sns+05, sfl+05, gsf+11}. Based on our 21 years 
timing of J1713+0747 along, we find 95\% confidence limits on the violations of SEP and
Lorentz invariance $\Delta < 0.01$ and $\hat{\alpha}_3<2\times10^{-20}$, 
slight improving the single pulsar limits from earlier data of this pulsar \citep{sfl+05, gsf+11}.
Stronger limits can be found by combining the results from
multiple similar pulsar systems \citep{wex00,sfl+05, gsf+11}.

%\section{Testing fundamental physics principles}
%\subsection{Conservation of momentum}
%\subsection{The Strong Equivalence Principle}

\section{Summary}
In this paper, we present a comprehensive model of high precision timing observations of
PSR~J1713+0747 that span 21 years. 
We improvement measurements on the pulsar and its companions' masses, the
geometry of the binary orbit. We also detect, for the first time, an apparent
decay of the orbit due to Galactic differential accelerations and the Shklovskii effect.
These measurements, when combined with those of other pulsars, 
significantly improve the pulsar timing limit on the change rate of the gravitational
 constant, $\dot{G}$, making it nearly as good as the best
 constraint of this kind (LLR experiment; \citep{hmb10}).
The new best pulsar timing limit on $\dot{G}/G$ is 
$0.4\pm11\times10^{-12}$~yr$^{-1}$ ($<0.023H_0$ based on 3-$\sigma$ limit), where $H_0$ is the Hubble constant. 
In other words, the change rate of gravitational constant has to be a factor
of at least $34$ (3-$\sigma$ limit) slower than the average expansion rate of
the Universe.
%Although this $\dot{G}/G$ limit is derived in the
%framework of scalar-tensor theories, it is a more rigorous and conservative
%limit than the generic phenomenological limit 
%$\dot{G}/G\simeq-\dot{P}_{\rm b}^{\rm exc}/(2P_{\rm
%b})=4\pm7\times10^{-13}$~yr$^{-1}$ \citep{dgt88}.

%Because $\dot P_{\rm b}$ is a second order term in the timing 
%model, its measured precision improves very fast, $\sigma \dot
%P_{\rm b} \propto T^{5/2}$, where $T$ is the total observational time
%span \cite{dt92}.
%It is possible that the precision of the above gravitational experiment from pulsar
%timing may surpass the precision of the LLR experiment in the future and
%become the best constraint of this type.

Meanwhile, the precise measurements of the PSR J1713+0747's orbit eccentricity and
3-D orientation allow us to test the violation of SEP and 
Lorentz invariance with it. We found a single-pulsar 95\% upper limit on 
$\Delta <0.01$, the SEP violation factor, and
$\hat{\alpha}_3<2\times10^{-20}$, the PPN parameter that characterize Lorentz
violation. 
Because of the different statistical analysis methods were used, our PSR
$\Delta$ and $\hat{\alpha}_3$ limits  are slightly
different but still consistent with the results of the same tests in previous publications 
\citep{wex00, sns+05, sfl+05, gsf+11}.
Ultimately, the best test on SEP violation could come from the newly
discovered pulsar triple system PSR J0337+1715 \citep{rsa+14}. In this case 
the inner pulsar-white dwarf binary is orbited by another white dwarf in a
outer orbit, making this system a great laboratory for testing
the free fall of neutron star and white dwarf in external gravity field.

We studied the time variation of PSR~J1713+0747's DM from 1998 to 2013, and found that 
the structure function of the DM variation followed a power law of 
index 0.63(7), which is significantly smaller than the 5/3 index expected from a ``pure''
Kolmogorov medium. We speculate this unusually flat DM structure function may
be partly caused by the event of large DM variation happened in 2008. This
2008 event might have increased the average DM variation in timescale of
hundreds of days, making the structure function appear flatter than expected.

Timing residuals of MSPs are believed to contain noises from timing
measurements, pulse phase jitter, and spin irregularities, etc. 
We found that the level of our timing residuals is consistent with the jitter
noise estimates from \citet{sc12} and the timing noise estimate from \citet{sc10}. However,
the scale law extrapolate from large sample study of timing noises in \citet{hlk10}
overestimate the timing noise level $\sigma_z({\rm 10 yr})$ in this pulsar.

%We measured the growth of timing residuals with time, and found no evidence of
%time-accumulative noises. This indicates that, unlike other younger pulsars,
%PSR~J1713+0747's timing residual consists of mostly white noise. Understanding the nature 
%of the noises in these MSPs is important for making predictions of detecting
%GWs using the pulsar timing array. Because ``red'' timing noise in MSPs reduces
%the sensitivity of the pulsar timing array to GWs.

In conjunction with the timing modeling, we also modeled observational noises
such as jitter noises and red noises
 using the general least square fit and a covariance matrix that included the
correlated and uncorrelated noise terms.
Our noise model parameters and timing residual RMS (Table \ref{tab:wrms})
provide a crude estimation of the amount of noises in our data. The WRMS of
the 21 years daily-averaged timing residuals is $\sim 118$~ns. 
%The smallest RMS values are from the 1400-MHz GUPPI data ($\sim$127~ns) and the
%2300-MHz PUPPI data ($\sim$134~ns),
%clearly better than thos from the older generations of instruments 
Table \ref{tab:wrms} show a systematic improvement in the timing accuracy of
this pulsar in the last two decades, due to the advances in instrumentations.
%, although not by as large as factor as implied
%by the radiometer equation, perhaps because of pulse jitter. 
Figure \ref{fig:rednoise} shows the reconstructed red noise signal with a
power law spectrum of a amplitude 0.06 ~$\mu$s at fiducial frequency of $1/$year 
and a spectrum index of -2.17. The significance of the red noise signal is consistent 
with the extrapolation from \citet{sc10}. The best-fit spectrum index favors
the idea that the spin irregularity was caused by random perturbations in
the spin down rate of the pulsar, but it does not exclude other explanations
due to its large uncertainty (see Figure \ref{fig:rednoise} for the
posterior distribution of the power law index).
Our reconstructed red noises are most significant in the Mark~IV and ABPP data
but not significant in the ASP, GASP, GUPPI, and PUPPI data.
This is consistent with the non-detection of red noises in the 5 year 
of ASP and GASP data by \citet{pjl+13} using an auto-correlation technique.



