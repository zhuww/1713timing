
\clearpage
\begin{deluxetable}{lccc}

\tabletypesize{\scriptsize}
\tablewidth{0pt}
\tablecaption{\label{tab:par2} Timing model parameters\tablenotemark{a} from {\it tempo2}. }
\tablehead{ \colhead{Parameter}  &\colhead{EFAC \& EQUAD}  &\colhead{with jitter model}  &\colhead{jitter \& red noise model}   }
\startdata
\textit{Measured Parameters}&  &  &  \\
Right Ascension, $\alpha$ (J2000)&  17:13:49.5320252(5)&  17:13:49.5320247(7)&  17:13:49.5320261(10)\\
Declination, $\delta$ (J2000)&  7:47:37.506130(13)&  7:47:37.506130(19)&  7:47:37.50615(3)\\
Spin Frequecy $\nu$~(s$^{-1}$)&  218.81184385472582(6)&  218.81184385472589(10)&  218.8118438547255(5)\\
Spin down rate $\nu'$ (s$^{-2}$)&  $-4.083889(5)\times10^{-16}$&  $-4.083892(7)\times10^{-16}$&  $-4.08386(5)\times10^{-16}$\\
Proper motion in $\alpha$, $\nu_{\alpha}=\dot{\alpha}\cos \delta$ (mas~yr$^{-1}$)&  4.9171(12)&  4.9181(18)&  4.915(3)\\
Proper motion in $\delta$, $\nu_{\delta}=\dot{\delta}$ (mas~yr$^{-1}$)&  -3.915(2)&  -3.910(3)&  -3.914(5)\\
Parallax, $\pi$ (mas)&  0.856(16)&  0.83(3)&  0.87(3)\\
Dispersion Measure\tablenotemark{b} (pc~cm$^{-3}$)&  15.97005(13)&  15.97006(12)&  15.97007(13)\\
Orbital inclination, $i$ (deg)&  75.1(3)&  71.9(7)&  71.9(7)\\
Companion Mass, $M_c$ ($M_{\odot}$)&  0.237(4)&  0.286(13)&  0.286(12)\\
Orbital Period, $P_{\rm b}$ (day)&  67.8251383100(12)&  67.8251383194(16)&  67.8251383185(17)\\
Change rate of $P_{\rm b}$, $\dot{P}_{\rm b}$ ($10^{-12}$s~s$^{-1}$)&  0.23(13)&  0.39(16)&  0.36(17)\\
Eccentricity, $e$&  0.0000749395(3)&  0.0000749400(6)&  0.0000749402(6)\\
Time of periastron passage, $T_0$ (MJD)&  53761.03232(11)&  53761.0327(3)&  53761.0328(3)\\
Angle of periastron\tablenotemark{c}, $\omega$ (deg)&  176.1943(6)&  176.1963(15)&  176.1966(14)\\
Projected semi-major axis, $x$ (lt-s)&  32.34242239(5)&  32.34242189(13)&  32.34242187(13)\\
Position angle of ascending node, $\Omega$ (deg)&  78.8(14)&  89.6(20)&  88(2)\\
Profile frequency dependency parameter, FD1 &  -0.0001632(2)&  -0.0001628(2)&  -0.0001628(2)\\
Profile frequency dependency parameter, FD2 &  0.0001357(3)&  0.0001355(3)&  0.0001355(3)\\
Profile frequency dependency parameter, FD3 &  -0.0000664(6)&  -0.0000671(6)&  -0.0000672(6)\\
Profile frequency dependency parameter, FD4 &  0.0000147(4)&  0.0000154(4)&  0.0000155(4)\\
\textit{Fixed Parameters}&  &  &  \\
Solar system ephemeris&  DE421&  DE421&  DE421\\
Reference epoch for $\alpha$, $\delta$, and $\nu$ (MJD)&  53729&  53729&  53729\\
Rate of periastron advance, $\dot{\omega}$ (deg/yr)\tablenotemark{d}&  0.00024&  0.00024&  0.00024\\
Red Noise Amplitude ($\mu$s/${
m yr}^{-1/2}$)&  --&  --&  --\\
Red Noise Spectral Index&  --&  --&  --\\
\textit{Derived Parameters}&  &  &  \\
Pulsar mass, $M_{\rm PSR}$ ($M_{\odot}$)&  1.00(3)&  1.31(11)&  1.31(11)\\
Dipole magnetic field, $B$ (G)&  $2.00\times10^{8}$&  $2.00\times10^{8}$&  $2.00\times10^{8}$\\
Characteristic age, $\tau_c$ (yr)&  $8.49\times10^{9}$&  $8.49\times10^{9}$&  $8.49\times10^{9}$
\enddata
\tablenotetext{a}{We used {\it tempo2}'s {\it T2} binary model, which
implicitly account for the change of the projected semi-major axis ($\dot{x}$
and allows us to fit for the position angle of ascending node ($\Omega$). Numbers in parentheses indicate the uncertainties on the last digit(s).  Uncertainties on parameters are estimated by the {\it tempo2} program using information in the covariance matrix.}
\tablenotetext{b}{The averaged DM value; See Section 3.2 and Figure 2 for more discussion.}
\tablenotetext{c}{See Figure 2 of \citealt{sns+05} for definition.}
\tablenotetext{d}{The rate of periastron advance was not fitted but fixed to the GR value
because it is higher co-variant with the orbital period.}


\end{deluxetable}

\clearpage 
